% =========================================================================== %

\begin{frame}[t,plain]
\titlepage
\end{frame}

% =========================================================================== %

\begin{frame}{Recap}
%
\begin{columns}[T]
\column{.5\linewidth}
\begin{itemize}
\item Klassen
	\begin{itemize}
	\item Datensammlung mit Kontext
	\item Klassen- und Instanz-Attribute
	\item Methoden: Funktionen, die auf Instanzen zugreifen
	\end{itemize}
\end{itemize}
%
\column{.5\linewidth}
\begin{itemize}
\item Dunders: Automatisch aufgerufene Methoden
	\begin{itemize}
	\item \inPy{__init__} -- Konstruktor
	\item \inPy{__str__} und \inPy{__repr__} -- Textdarstellungen
	\item \inPy{__call__} -- Aufruf als Funktion
	\end{itemize}
\end{itemize}

\end{columns}
%
\begin{center}
	\emph{Noch Fragen?}
\end{center}
%
\end{frame}

% =========================================================================== %

\begin{frame}[fragile]{Kapitel 7}
%
\begin{itemize}
\item Dunders: 
	\begin{itemize}
	\item Vergleiche
	\item Rechenoperationen
	\item Rechtsseitige Operatoren
	\item Index-Zugriff
	\item Zahl der Elemente, Absolutbetrag
	\end{itemize}
\item Vererbung
\end{itemize}
%
\end{frame}

% =========================================================================== %

\begin{frame}[fragile]{\inPy{__len__} und \inPy{__abs__}}
%
\begin{columns}[T]
\column{.5\linewidth}
\begin{itemize}
\item Werden aufgerufen von \inPy{len(Instanz)} bzw. \inPy{abs(Instanz)}
\item Bedeutung frei wählbar, sollte aber die Eigenschaften wiederspiegeln, die man sich unter Absolutbetrag und Länge vorstellt
\end{itemize}
%
\column{.5\linewidth}
\begin{codebox}[Beispiel: \texttt{\_\_abs\_\_}]
\begin{minted}[fontsize=\scriptsize, linenos]{python3}
class Expartner :
    # Attribute und Definitionen wie
    # beim letzten Mal
    
    def __abs__(self) :
        retrun self.rating()
    
    def __len__(self) :
        return self.height

print(f"Zeit mit {ex.name}: {abs(ex)}")
print(f"Größe von {ex.name}: {len(ex)}")
\end{minted}
\end{codebox}
\end{columns}
%
\end{frame}

% =========================================================================== %

\begin{frame}[fragile]{\inPy{__eq__}, \inPy{__ne__}, \inPy{__gt__}, \inPy{__ge__}, \inPy{__lt__} und \inPy{__le__}: Vergleiche}
%
\begin{columns}[T]
\column{.5\linewidth}
\begin{itemize}
\item Funktionen, die entscheiden, in welchem Verhältnis zwei Instanzen zueinander stehen
\item Aufruf über Vergleichsoperator
\item Vergleichsoperator: \texttt{==}, \texttt{!=}, >, \texttt{>=}, \texttt{<}, \texttt{<=}
\item Beispiel: \inPy{Instanz1 == Instanz2} 
\item[\Thus] \inPy{Instanz1.__eq__(Instanz2__)}
\end{itemize}
%
\column{.5\linewidth}
\begin{codebox}[Syntax: Definition]
\begin{minted}[fontsize=\scriptsize]{python3}
class Klassenname :
    def __XX__ (self, rhs) :
       return ... # True, wenn self XX rhs
\end{minted}
\end{codebox}
%
\begin{codebox}[Syntax: Aufruf]
\begin{minted}[fontsize=\scriptsize]{python3}
print(Instanz1 XX Instanz2)
\end{minted}
\end{codebox}
\end{columns}
%
\end{frame}

% =========================================================================== %

\begin{frame}[fragile]
%
\begin{codebox}[Beispiel: Direkter Vergleich von Instanzen]
\begin{minted}[linenos, fontsize=\scriptsize]{python3}
class Expartner :
    # __init__, Klassenattribute, Methode rating wie letztes Mal
    
    def __gt__(self, rhs) :
        return self.rating() > rhs.rating()
    
    def __le__(self, rhs) :
        return not (self < rhs)

# Definition Instanzen ex1, ex2 wie letzte Woche

betterText = "nicht" if ex1 <= ex2 else ""
print(f"Mit {ex1.name} war es {betterText} besser als mit {ex2.name}.")
\end{minted}
\end{codebox}
%
\end{frame}

% =========================================================================== %

\begin{frame}[fragile]
%
\begin{hintbox}[Tipp: Sortierbare Listen]
Sobald eine Klasse entweder \inPy{__lt__} oder \inPy{__gt__} implementiert, ist es möglich, \inPy{sorted} bzw. \inPy{listOfInstances.sort()} anzuwenden!
\end{hintbox}
%
\begin{codebox}[Beispiel: Listen Sortieren]
\begin{minted}[linenos, fontsize=\scriptsize]{python3}
# Klasse Expartner wie vorhin
# Einige Instanzen von Expartner mit Namen ex1, ex2, ...

listOfExes = [ex1, ex2, ex3, ex4]

for i, ex in enumerate(sorted(listOfExes, reverse=True)) :
    print(f"Rank #{i+1}: {ex.name}")
\end{minted}
\end{codebox}
%
\end{frame}

% =========================================================================== %

\begin{frame}[fragile]{Typensicherheit}
%
\begin{itemize}
\item Meiste Sprachen: Datentyp in Parameterliste vorgegeben.
\item Python: \emph{Dynamic Typing} -- alles geht
\item Resultat: syntaktisch korrekte, aber unsinnige Aufrufe
\end{itemize}
%
\begin{tcbraster}[raster columns=2,
                  raster equal height,
                  nobeforeafter,
                  raster column skip=0.5cm]
\begin{codebox}[Beispiel: Unsinniger Aufruf]
\begin{minted}[fontsize=\scriptsize, linenos]{python3}
# Klasse Expartner wie zuvor
# Instanz MyEx

print( myEx > 7 )
# Gleichbedeutend mit myEx.__gt__(7)
\end{minted}
\end{codebox}
%
\begin{cmdbox}[Ausgabe: Unsinniger Aufruf]
\begin{minted}[fontsize=\scriptsize]{text}
Traceback (most recent call last):
  File "exes.py", line 289, in <module>
    print(allExes[0] < 0)
TypeError: '<' not supported between
  instances of 'Expartner' and 'int'
\end{minted}
\end{cmdbox}
\end{tcbraster}
%
\end{frame}

% =========================================================================== %

\begin{frame}[fragile]{Typensicherheit manuell einbringen}
%
\begin{codebox}[Beispiel: Typensicherer Vergleich]
\begin{minted}[fontsize=\scriptsize, linenos]{python3}
class Expartner :
    # alles wie immer
    
    def __gt__ (self, rhs) :
        if type(rhs) in (int, float) :
            return self.rating() > rhs
        elif type(rhs) == Expartner :
            return self.rating() > rhs.rating()
        else :
            raise Exception(f"Typen 'Expartner' und '{type(rhs)}'nicht vergleichbar")
            # Löst Fehlermeldung aus und beendet Programm
            # Mehr dazu in Kapitel 9

print( myEx > 7 )  # funktioniert!
\end{minted}
\end{codebox}
%
\end{frame}

% =========================================================================== %

\begin{frame}[fragile]{\inPy{__add__}, \inPy{__neg__}, \inPy{__sub__}, \inPy{__mul__}, \inPy{__truediv__}, \inPy{__floordiv__} und \inPy{__matmul__}: Rechenoprationen}
%
\begin{columns}[T]
\column{.5\linewidth}
\begin{itemize}
\item Logik wie bei Vergleichsoperationen
\item Implementieren Operatoren \inPy{A + B}, \inPy{-A}, \inPy{A - B}, \inPy{A * B}, \inPy{A / B}, \inPy{A // B} und\\
 \texttt{A @ B}
\item Operator \texttt{@}: \emph{frei wählbare Verknüpfung}, üblicherweise aber Matrixprodukt
\end{itemize}
%
\column{.5\linewidth}
\begin{codebox}[Syntax: Definition]
\begin{minted}[fontsize=\scriptsize]{python3}
class Klassenname :
    def __XX__ (self, rhs) :
       return ...
\end{minted}
\end{codebox}
%
\begin{codebox}[Syntax: Aufruf]
\begin{minted}[fontsize=\scriptsize]{python3}
print(Instanz1 XX Instanz2)
\end{minted}
\end{codebox}
\end{columns}
%
\vspace{6pt}
\emph{Im Weiteren: Vektoren. Ich weiß nicht welche Bedeutung \texttt{Ex1 // Ex2} haben könnte...}
%
\end{frame}

% =========================================================================== %

\begin{frame}
%
\begin{hintbox}[Tipp: Operatoren durch ihre Gegenstücke definieren]
Es kann aufwändig, zumindest aber lästig werden, die sehr ähnlichen Operationen für eine Klasse zu definieren. Wir finden oft redundanten Code, der sich nur durch Kleinigkeiten unterscheidet. Bei Operatoren können wir das umgehen, indem wir nur einige wenige Operationen voll implementieren. Der Rest wird durch ihr gegenteil ausgedrückt: \inPy{A >= B} ist \inPy{not A < B} und \inPy{A - B} ist \inPy{A + (-B)} (\inPy{__neg__} und \inPy{__add__})
\end{hintbox}
%
\end{frame}

% =========================================================================== %

\begin{frame}
%
\begin{center}
	(Code: \texttt{004a-vectors.py})
\end{center}
%
\end{frame}

% =========================================================================== %

\begin{frame}[fragile]{Links- und Rechtsseitige Operatoren}
%
Problem erkannt: Multiplikation \emph{von links} wird der Klasse \inPy{int} zugeordnet:
\begin{warnbox}[Beispiel: Title goes here, leftupper=6mm]
\begin{minted}[linenos, firstnumber=90, fontsize=\scriptsize]{python3}
print("Reskalieren  :", v * 2)
print("Rechtseitig  :", 2 * v)
\end{minted}
\end{warnbox}
%
\begin{cmdbox}[Ausgabe: Unsinniger Aufruf]
\begin{minted}[fontsize=\scriptsize]{text}
Traceback (most recent call last):
  File "004a-vectors.py", line 91, in <module>
    print("Rechtseitig  :", 2 * v)
TypeError: unsupported operand type(s) for *: 'int' and 'vector3D'
\end{minted}
\end{cmdbox}
%
\end{frame}

% =========================================================================== %

\begin{frame}[fragile]{\inPy{NotImplemented} und Rechtsseitige Operatoren}
%
\begin{itemize}
\item Besonderer Rückgabewert: \inPy{NotImplemented}
\item Wird von Python abgefangen. Je nach Szenario: Verschiedene Strategien
\item Hier: Versuch, die Methode \inPy{__rmul__} aufzurufen
	\begin{itemize}
	\item Wie \inPy{__mul__}, aber geht davon aus dass \inPy{self} auf der \emph{rechten} Seite steht
	\item[\Thus] \texttt{r} in \texttt{rmul}
	\end{itemize}
\item Ebenso: \inPy{__radd__}, \inPy{__rsub__}, ...
\item Wenn diese Methoden nicht existieren: Fehlermeldung und Abbruch
\end{itemize}
%
\end{frame}

% =========================================================================== %

\begin{frame}[fragile]
%
\begin{codebox}[Beispiel: Multiplikation von links ermöglichen]
\begin{minted}[linenos, fontsize=\scriptsize]{python3}
class Vector3D :
    # alles wie immer
    
    # ........................................................................ #
    
    def __mul__(self, rhs) :
        if type(rhs) == vector3D :
            return self.x * rhs.x  +  self.y * rhs.y  +  self.z * rhs.z
        elif type(rhs) in (int, float) :
            return vector3D(self.x * rhs, self.y * rhs, self.z * rhs)
        else :
            return NotImplemented   #  <== Python Alternative suchen lassen
    
    # ........................................................................ #
    
    def __rmul__(self, lhs) :       #  <== und Alternative anbieten
      return self * lhs;

print(2 * v)    # ruft jetzt __rmul__(v, 2) auf!
\end{minted}
\end{codebox}
%
\end{frame}

% =========================================================================== %

\begin{frame}[fragile]{Index-Zugriff}
%
\begin{columns}[T]
\column{.5\linewidth}
\begin{itemize}
\item Szenario: Zugriff über Index gewünscht. Z.\,B. \inPy{v[1]} \thus \inPy{v.x}
\item Tatsächlich nützlich, da so Schleifen leichter machbar
\item Oft auch: Unsinnige Zugriffe oder Werte verbieten
\item Lösung über zwei neue Dunders:
	\begin{itemize}
	\item \inPy{__getitem__(self, index)} \\ Lesezugriff
	\item \inPy{__setitem__(self, index, value)} \\ Schreibzugriff
	\end{itemize}
\item \texttt{index} ist dabei der Ausdruck in den \texttt{[}eckigen Klammern\texttt{]}
\end{itemize}
%
\column{.5\linewidth}
\begin{codebox}[Syntax: Title goes here]
\begin{minted}[fontsize=\scriptsize]{python3}
class Klassenname :
    ...
    
    def __getitem__(self, index) :
        return ...
    
    def __setitem__(self, index, value) :
        ... = value
\end{minted}
\end{codebox}
%
\begin{itemize}
\item \texttt{index} kann beliebige Datentypen haben!
\item Vgl. \inPy{dict}s!
\item \inPy{Instanz[1, 2]} \Thus \inPy{index = (1, 2)}, also ein \inPy{tuple}
\end{itemize}
\end{columns}
%
\end{frame}

% =========================================================================== %

\begin{frame}[fragile]
%
\begin{codebox}[Beispiel: Index-Getter und -Setter]
\begin{minted}[linenos, fontsize=\scriptsize]{python3}
class Vector3D :
    # alles wie immer
    # ........................................................................ #
    def _checkIndex(self, index) :
        if type(index) != int :
            raise Exception("Index-Typ nicht unterstützt!")
        elif index not in (1, 2, 3) :
            raise Exception("Index nicht im zulässigen Bereich")
    # ........................................................................ #    
    def __getitem__(self, index) :
        self._checkIndex(index)
        if   index == 1 : return self.x
        elif index == 2 : return self.y
        elif index == 3 : return self.z    
    # ........................................................................ #    
    def __setitem__(self, index, value) :
        self._checkIndex(index)
        if   index == 1 : self.x = value
        elif index == 2 : self.y = value
        elif index == 3 : self.z = value
\end{minted}
\end{codebox}
%
\end{frame}

% =========================================================================== %

\begin{frame}[fragile]
%
\begin{hintbox}[Tipp: Gemeinsamkeiten in separater Methode]
Wo zumindest Gemeinsamkeiten zwischen Methoden bestehen, kann eine \enquote{versteckte Methode} die gemeinsame Vorarbeit leisten.

Oft werden Funktionen \enquote{versteckt}, indem ein Unterstrich (\texttt{\_}) davor gesetzt wird.
\end{hintbox}
%
\begin{codebox}[Schema: Aufteilung in geteilte und spezielle Arbeitsblöcke]
\begin{minted}[linenos, fontsize=\scriptsize]{python3}
def _gemeinsamerCode(gemeinsameParameter) :
    ...

def procA(gemeinsameParamter, spezifischeParameter) :
   _gemeinsamerCode(gemeinsameParamter)
   spezifischer Code A
   ...

def procB(gemeinsameParamter, spezifischeParameter) :
   _gemeinsamerCode(gemeinsameParamter)
   spezifischer Code B
   ...
\end{minted}
\end{codebox}
%
\end{frame}

% =========================================================================== %

\begin{frame}[fragile]{Vererbung}
%
\begin{columns}[T]
\column{.5\linewidth}
\begin{itemize}
\item Idee: neue Klasse erstellen, die eine Erweiterung einer bestehenden Klasse ist
\item Sprechweise: \emph{Basisklasse} oder \emph{parent class}
	\begin{itemize}
	\item Vorlage, wird kopiert
	\end{itemize}
\item Sprechweise: \emph{abgeleitete Klasse} oder \emph{derived class}, \emph{child}
	\begin{itemize}
	\item Neu angelegt
	\item Erhält alle Attribute und Methoden der Basisklasse
	\item Darf beliebige neue Definitionen erhalten
	\item \enquote{abgeleitete Klasse erbt von Basisklasse}
	\end{itemize}
\end{itemize}
%
\column{.5\linewidth}
\begin{codebox}[Syntax: Abgeleitete Klasse erstellen]
\begin{minted}[fontsize=\scriptsize]{python3}
class Abgeleitet(BasisKlasse) :
   # beliebiger Code
\end{minted}
\end{codebox}
%
\begin{itemize}
\item Eigenschaften der Basisklasse können überschrieben werden
\item Ändert sich dann nur in der abgeleiteten Klasse
\end{itemize}
\end{columns}
%
\end{frame}

% =========================================================================== %

\begin{frame}[fragile]
%
\begin{codebox}[Beispiel: Abgeleitete Klasse]
\begin{minted}[linenos, fontsize=\scriptsize]{python3}
class Vector3D :
    # ... wie vorhin

class ColourArrow (Vector3D) :
    Arrowcount = 0    # neues Klassenattribut
    
    def __init__(self, x, y, z, color) : # überschreibt / ersetzt alte Methode
        self.x = x                       # für abgeleitete Klasse
        self.y = y
        self.z = z
        self.color = color
        ColourArrow.Arrowcount += 1
    
    def __str__(self) :
        # ebenfalls: Überschreiben
        return "(" + str(self.x) + ", " + str(self.y) + ", " + str(self.z) + \
               ") in " + self.color

cA = ColourArrow(1, 4, 9, "red")      # Aufruf neue Methode
print( abs(cA) )                      # Aufruf alte Methode
\end{minted}
\end{codebox}
%
\end{frame}

% =========================================================================== %

\begin{frame}[fragile]{Bezüge zur Basisklasse -- \inPy{super()}}
%
\begin{itemize}
\item In gezeigter Form: Copy\&Paste-Code von der Basisklasse
\item Stattdessen: Aufruf von Code der Basisklasse
\item \inPy{super()}: \enquote{Berechnet} ein Objekt, das formal zur Basisklasse gehört, aber \inPy{self} beschreibt
\item[\Thus] Methoden der Basisklasse auf \inPy{self} anwendbar
\end{itemize}
%
\end{frame}

% =========================================================================== %

\begin{frame}[fragile]
%
\begin{codebox}[Beispiel: Abgeleitete Klasse mit \texttt{super()}]
\begin{minted}[linenos, fontsize=\scriptsize]{python3}
class Vector3D :
    # ... wie vorhin

class ColourArrow (Vector3D) :
    Arrowcount = 0    # neues Klassenattribut
    
    def __init__(self, x, y, z, color) :
        super().__init__(x, y, z)
        self.color = color
        ColourArrow.Arrowcount += 1
    
    def __str__(self) :
        return super().__str__() + " in " + self.color

cA = ColourArrow(1, 4, 9, "red")
print( abs(cA) )
print("Es gibt", ColourArrow.Arrowcount, "ColourArrows.")
print(cA)
\end{minted}
\end{codebox}
%
\end{frame}

% =========================================================================== %

\begin{frame}[fragile]{Multi-Vererbung und Vererbungs-Ebenen}
%
\begin{itemize}
\item abgeleitete Klasse kann selbst wieder Basisklasse sein \Thus mehrere Ebenen möglich
\item abgeleitete Klasse kann auch Eigenschaften von \emph{mehreren} Basisklassen übernehmen
\item \inPy{class abgeleiteteKlasse (Basisklasse1, Basisklasse2, ...) :}
\item Darf nicht zu Namenskollision führen
	\begin{itemize}
	\item[\Thus] Wenn \texttt{Basisklasse1} schon \texttt{Methode} implementiert, darf \texttt{Basisklasse2} das nicht auch tun.
	\item[\Thus] Falls doch: nur \texttt{Methode} von erstgenannter \texttt{Basisklasse1} wird übernommen
	\end{itemize}
\end{itemize}
%
\end{frame}

% =========================================================================== %

\begin{frame}[fragile]{Klassendiagramm}
%

\tikzstyle{class}=[rectangle, draw=black, rounded corners, fill=blue!40!black,
        text centered, anchor=north, text=white, text width=4.5cm]
\tikzstyle{instance}=[rectangle, draw=black, rounded corners, fill=green!40!white,
        text centered, anchor=north, text=black, text width=4.5cm]
                
\tikzstyle{inherits}=[->, >=open triangle 90, thick]
\tikzstyle{instantiates}=[->, >=open diamond, thick]
%
\begin{tikzpicture}[node distance=1cm]
	\node (Animal) [class]                  {\texttt{Animal}};
	\node (Mammal) [class, below=of Animal] {\texttt{Mammal}};
	
	\node (AuxNode01) [          below=of Mammal   , text width=0.0cm] {};
	\node (AuxNode02) [          right=of Animal   , text width=0.5cm] {};
	\node (Klasse)    [class,    right=of AuxNode02, text width=3.0cm] {\texttt{Klasse}};
	\node (Instanz)   [instance, below=of Klasse   , text width=3.0cm] {\texttt{Instanz}};
	
	\node (NonWingedMammal) [class, left =of AuxNode01] {\texttt{NonWingedMammal}};
	\node (NonMarineMammal) [class, right=of AuxNode01] {\texttt{NonMarineMammal}};
	
	\node (Dog) [class, below=of AuxNode01] {\texttt{Dog}};
	
	\node (d)   [instance, below=of Dog] {\texttt{d}};
	\node (bat) [instance, right=of d]   {\texttt{bat}};
	
	\draw[inherits] (Mammal.north) -| (Animal.south);
	\draw[inherits] (NonWingedMammal.north) -- ++(0,0.36) -| (Mammal.south);
	\draw[inherits] (NonMarineMammal.north) -- ++(0,0.40) -| (Mammal.south);
	\draw[inherits] (Dog.north) -- ++(0,0.40) -| (NonMarineMammal.south);
	\draw[inherits] (Dog.north) -- ++(0,0.40) -| (NonWingedMammal.south);
	
	\draw[instantiates] (d.north)   -- ++(0,0.40) -| (Dog.south);
	\draw[instantiates] (bat.north) -- ++(0,1.20) -| ([xshift=0.8cm] NonMarineMammal.south);
\end{tikzpicture}
%
\end{frame}
% =========================================================================== %

\begin{frame}[fragile]
%
\begin{tcbraster}[raster columns=2,
                  raster equal height,
                  nobeforeafter,
                  raster column skip=0.5cm]
\begin{codebox}[Beispiel: Multi-Vererbung ...]
\begin{minted}[fontsize=\scriptsize, linenos]{python3}
class Animal :
  def __init__(self, Animal) :
    print(Animal, 'is an animal.')
    
class Mammal(Animal) :
  def __init__(self, mammalName) :
    print(mammalName, 
          'is a warm-blooded animal.')
    super().__init__(mammalName)

class NonWingedMammal(Mammal) :
  def __init__(self, NonWingedMammal) :
    print(NonWingedMammal, "can't fly.")
    super().__init__(NonWingedMammal)
\end{minted}
\end{codebox}
%
\begin{codebox}[... Fortsetzung]
\begin{minted}[fontsize=\scriptsize, linenos, firstnumber=last]{python3}
class NonMarineMammal(Mammal) :
  def __init__(self, NonMarineMammal) :
    print(NonMarineMammal, 
          "can't swim.")
    super().__init__(NonMarineMammal)
    
class Dog(NonMarineMammal,
          NonWingedMammal) :
  def __init__(self) :
    print('Dog has 4 legs.')
    super().__init__('Dog')

d = Dog()
print('')
bat = NonMarineMammal('Bat')
\end{minted}
\end{codebox}
\end{tcbraster}
%
\begin{center}
	\scriptsize
	Quelle:
	\url{https://www.programiz.com/python-programming/methods/built-in/super}
\end{center}
%
\end{frame}

% =========================================================================== %

\begin{frame}[fragile]
%
\begin{tcbraster}[raster columns=2,
                  raster equal height,
                  nobeforeafter,
                  raster column skip=0.5cm]
\begin{cmdbox}[Ausgabe: Multi-Vererbung]
\begin{minted}[fontsize=\scriptsize]{text}
Dog has 4 legs.
Dog can't swim.
Dog can't fly.
Dog is a warm-blooded animal.
Dog is an animal.

Bat can't swim.
Bat is a warm-blooded animal.
Bat is an animal.
\end{minted}
\end{cmdbox}
%
\begin{hintbox}[Tipp: Klassenbeziehungen einfach halten]
In der Praxis führt Vererbung über mehrere Ebenen und vor allem Multi-Vererbung zu schwer durchschaubaren Konstrukten.
Vermeiden Sie komplexe Gebilde, und halten Sie Ihre Klassen so autark wie möglich.
\end{hintbox}
\end{tcbraster}
%
\end{frame}