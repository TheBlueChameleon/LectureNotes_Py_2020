% =========================================================================== %

\begin{frame}[t,plain]
\titlepage
\end{frame}

% =========================================================================== %

\begin{frame}{Recap}
%
\begin{columns}[T]
\column{.5\linewidth}
\begin{itemize}
\item Kommandozeile
	\begin{itemize}
	\item Text-Befehle mit Parametern
	\item Arbeitsverzeichnis
	\item Python-Interpreter aufrufen
	\end{itemize}
\item Befehl \inPy{print}
	\begin{itemize}
	\item Text-Ausgabe
	\item Argumente in Klammern, durch Kommata getrennt
	\item Texte in Anführungszeichen
	\end{itemize}
\end{itemize}
%
\column{.5\linewidth}
\begin{itemize}
\item Variablen
	\begin{itemize}
	\item Speicherstellen mit veränderlichem Inhalt
	\item Bezeichner und Datentyp
	\item Rechnen und Updaten
	\item Verwendung mit \inPy{print}
	\end{itemize}
\end{itemize}
\end{columns}
%
\begin{center}
	\emph{Noch Fragen?}
\end{center}
%
\end{frame}

% =========================================================================== %

\begin{frame}[fragile]{Kommentare}
%
\begin{minipage}[t]{.44\linewidth}
\begin{itemize}
\item Text-Teile, die vom Interpreter ignoriert werden
\item Erinnerungen und Erklärungen für uns als ProgrammiererInnen
\item Zeilen zum Testen schnell ein- und ausschalten
\item Beginnen mit Raute (\texttt{\#})
\item Gehen bis zum Ende der Zeile
\end{itemize}
\end{minipage}
%
\begin{minipage}[t]{.55\linewidth}
\phantom{x}
\begin{codebox}[Code mit Kommentaren]
\begin{minted}[linenos, fontsize=\scriptsize]{python}
print("Hello World")  # Textausgabe

# Auch als eigene Zeile

# Verschachtelte # Kommentare existieren nicht

# print("Code, der nicht ausgeführt wird")
\end{minted}
\end{codebox}
\end{minipage}
%
\end{frame}

% =========================================================================== %

\begin{frame}[fragile]{Datentypen umwandeln}
%
\begin{itemize}
\item Variablen speichern Werte
\item Rechnen mit Werten im allgemeinen möglich.
\item Datentypen beeinflussen das Ergebnis (\inPy{"1" + "2"} vs. \inPy{1 + 2})
\item Inkompatible Datentypen (\inPy{"1" + 2} ergibt Fehlermeldung)
\item Datentypen als Funktion: \inPy{float(1)} \thus~ \inPy{1.0}
\end{itemize}
%
\begin{minipage}[t]{.52\linewidth}
\phantom{x}
\begin{codebox}[Code: Typen-Umwandlung]
\begin{minted}[linenos, fontsize=\scriptsize]{python}
text = "Das Ergebnis lautet: "
result = ((1 + 1j) * (7.5 / 19 + 1)) ** 2

text += str(result)

print(text)
\end{minted}
\end{codebox}
\end{minipage}
%
\begin{minipage}[t]{.47\linewidth}
\phantom{x}
\begin{cmdbox}[Ausgabe: Typen-Umwandlung]
\begin{minted}[fontsize=\scriptsize]{text}
Das Ergebnis lautet: 3.890581717451524j
\end{minted}
\end{cmdbox}
\end{minipage}
%
\end{frame}

% =========================================================================== %

\begin{frame}[fragile]{Datentypen umwandeln}
%
\begin{itemize}
\item Datenverlust möglich
	\begin{itemize}
	\item \inPy{int(1.2)} \thus~ \inPy{1}
	\end{itemize}
\item Nicht immer möglich
	\begin{itemize}
	\item \inPy{int("1")} -- okay, Ergebnis \inPy{1}
	\item \inPy{float("1.0")} -- okay, Ergebnis \inPy{1.0}
	\item  \inPy{int("1.0")} -- Fehlermeldung
	\end{itemize}
\end{itemize}
%
\begin{cmdbox}[Fehlermeldung -- \texttt{int("1.0")}]
\begin{minted}[fontsize=\scriptsize]{text}
Traceback (most recent call last):
  File "<stdin>", line 1, in <module>
ValueError: invalid literal for int() with base 10: '1.0'
\end{minted}
\end{cmdbox}
%
\end{frame}

% =========================================================================== %

\begin{frame}[fragile]{Formatierte Texte}
%
\begin{itemize}
\item Tabellarische Ausgaben
\item Zahl der Nachkommastellen nicht fix
\item Lösung: Spezielle Strings, die Format vorgeben
\item Präfix \inPy{f}, dann normaler String in Anführungszeichen(\inPy{"} oder \inPy{'})
\item Platzhalter für Variablenwerte in \{geschweiften Klammern\}
\item Optional: Doppelpunkt und Formatierungszeichen
\end{itemize}
%
\begin{minipage}[t]{.49\linewidth}
\phantom{x}
\begin{codebox}[Code: Formatierter Text (1)]
\begin{minted}[linenos, fontsize=\scriptsize]{python}
x = 7 / 3
text = f"x = {x}"

print(text)
\end{minted}
\end{codebox}
\end{minipage}
%
\begin{minipage}[t]{.49\linewidth}
\phantom{x}
\begin{cmdbox}[Ausgabe: Formatierter Text (1)]
\begin{minted}[fontsize=\scriptsize]{text}
x = 2.3333333333333335
\end{minted}
\end{cmdbox}
\end{minipage}
%
\end{frame}

% =========================================================================== %

\begin{frame}[fragile]{Formatierte Texte: Spaltenbreite}
%
\begin{itemize}
\item Formatierungszeichen: Im einfachsten Fall eine Zahl
\item Zahl der Schriftzeichen, die für diesen Ausdruck vorgesehen sind
\item Füllt mit Leerzeichen auf, druckt vollen Text
\item \emph{Keine} Leerzeichen
\end{itemize}
%
\vspace{-10pt}
\begin{minipage}[t]{.49\linewidth}
\phantom{x}
\begin{codebox}[Code: Formatierter Text (2)]
\begin{minted}[linenos, fontsize=\scriptsize]{python}
name1 = "Dusky"
score1 = 9001
name2 = "Joe"
score2 = 666

print("Highscore")
print(f"{'Python':20}: {123456:5}")
print(f"{name1:20}: {score1:5}")
print(f"{name2:20}: {score2:5}")
\end{minted}
\end{codebox}
\end{minipage}
%
\begin{minipage}[t]{.49\linewidth}
\phantom{x}
\begin{cmdbox}[Ausgabe: Formatierter Text (2)]
\begin{minted}[fontsize=\scriptsize]{text}
Highscore
Python              : 123456
Dusky               :  9001
Joe                 :   666
\end{minted}
\end{cmdbox}
\end{minipage}
%
\end{frame}

% =========================================================================== %

\begin{frame}[fragile]{Formatierte Texte: Bündigkeit und Beschneiden}
%
\begin{itemize}
\item \texttt{<}, \texttt{>}, \inPy{^}: Links- und Rechtsbüdnig, Zentriert
\item \texttt{.Zahl}: Auf [Zahl] Schriftzeichen begrenzen
\item Leerzeichen und Plus: Platz für Vorzeichen
\item \texttt{0}: Führende Nullen anzeigen
\item \emph{Keine} Leerzeichen
\end{itemize}
%
\vspace{-10pt}
\begin{minipage}[t]{.49\linewidth}
\phantom{x}
\begin{codebox}[Code: Formatierter Text (3)]
\begin{minted}[linenos, fontsize=\scriptsize]{python}
text = "sample"
value = 123
print( f"|{text:15}| |{value:15}|" )
print(f"|{text:<15}| |{value:<15}|")
print(f"|{text:>15}| |{value:>15}|")
print(f"|{text:^15}| |{value:^15}|")
\end{minted}
\end{codebox}
\end{minipage}
%
\begin{minipage}[t]{.49\linewidth}
\phantom{x}
\begin{cmdbox}[Ausgabe: Formatierter Text (3)]
\begin{minted}[fontsize=\scriptsize]{text}
|sample         | |            123|
|sample         | |123            |
|         sample| |            123|
|    sample     | |      123      |
\end{minted}
\end{cmdbox}
\end{minipage}
%
\end{frame}

% =========================================================================== %

\begin{frame}[fragile]{Formatierte Texte: Fließkommawerte}
%
\begin{itemize}
\item Formatierungszeichen \texttt{f}
\item \texttt{.Zahl}: Auf [Zahl] Schriftzeichen begrenzen
\item \texttt{Gesamtlaenge.Nachkommaanteil} -- Dezimalpunkt wird mitgezählt
\item Kombinierbar mit bekannten Zeichen
\end{itemize}
%
\vspace{-10pt}
\begin{minipage}[t]{.49\linewidth}
\phantom{x}
\begin{codebox}[Code: Formatierter Text (4)]
\begin{minted}[linenos, fontsize=\scriptsize]{python}
num = 1.2
print(f"|{num}|")
print(f"|{num:f}|")
print(f"|{num:6.2f}|")
print(f"|{num:<6.1f}|")
print(f"|{num:06.2f}|")
print(f"|{num:+06.2f}|")
print(f"|{num: 06.2f}|")
\end{minted}
\end{codebox}
\end{minipage}
%
\begin{minipage}[t]{.49\linewidth}
\scriptsize\phantom{x}
\begin{cmdbox}[Ausgabe: Formatierter Text (4)]
%\begin{minted}[fontsize=\scriptsize]{text}
\phantom{x}\\
|1.2|\\
|1.200000|\\
|~~1.20|\\
|1.2~~~|\\
|001.20|\\
|+01.20|\\
| 01.20|
%\end{minted}
\end{cmdbox}
\end{minipage}
%
\end{frame}

% =========================================================================== %

\begin{frame}{Die Große Frage}
%
\begin{center}
\begin{Large}
\emph{Muss ich das alles auswendig wissen?}
\end{Large}
\end{center}
%
\begin{itemize}
\item Nein.
\item Script, Folien, Internet, ... jederzeit zu eurer Verfügung
\item Wissen, dass es das gibt, und wo man nachsehen kann
\item Flüssig Lesen können
\item Mit Hilfsmitteln umsetzen können
\item Mit Praxis kommt das irgendwann von alleine
\item Nach 20 Jahren muss ich immer noch Details nachschlagen
\end{itemize}
%
\end{frame}

% =========================================================================== %

\begin{frame}{Kapitel 2}
%
\begin{itemize}
\item  Usereingaben und Entscheidungen
	\begin{itemize}
	\item Usereingaben -- \inPy{input}
	\item Booleans (Wahrheitswerte), Vergleichsoperatoren und logische Operatoren
	\item \inPy{if}-Blöcke
	\item Der Ternäre Operator
	\end{itemize}
\end{itemize}
%
\end{frame}

% =========================================================================== %

\begin{frame}[fragile]{Usereingaben}
%
\begin{itemize}
\item Daten Lesen von der Tastatur mit \inPy{input}.
\item Parameter \enquote{Prompt}: Text, der als \enquote{Frage} gestellt wird
\item Rückgabe der Eingabe als String
\item Gegebenenfalls Typenumwandlung nötig.
\item User kann (und wird) Unsinn eingeben!
\end{itemize}
%
\vspace{-10pt}
\begin{minipage}[t]{.49\linewidth}
\phantom{x}
\begin{codebox}[Code: Dateneingabe]
\begin{minted}[linenos, fontsize=\scriptsize]{python}
inText = input(
    "Bitte geben Sie eine Zahl ein: ")
inNumber = float(inText)
print("Das Quadrat dieser Zahl ist",
      inNumber ** 2)
\end{minted}
\end{codebox}
\end{minipage}
%
\begin{minipage}[t]{.49\linewidth}
\phantom{x}
\begin{cmdbox}[Ausgabe: Dateneingabe]
\begin{minted}[fontsize=\scriptsize]{text}
Bitte geben Sie eine Zahl ein: 1.5
Das Quadrat dieser Zahl ist 2.25
\end{minted}
\end{cmdbox}
\end{minipage}
%
\end{frame}

% =========================================================================== %

\begin{frame}[fragile]{Wahrheitswerte}
%
\begin{itemize}
\item Datentyp \inPy{bool}: Gedanke Ja (\inPy{True}) oder Nein (\inPy{False})
\item Ergebnis von Vergleichen: \inPy{1 + 5 * 8 + 1 == 42} ist \inPy{True}
\item Kann in Variablen gespeichert werden: \inPy{truth = (1 + 5 * 8 + 1 == 42)}
\item Intern: Zahlen \inPy{1} und \inPy{0} mit spezieller Interpretation.\\
	\inPy{True + True == 2}
\end{itemize}
%
\newcolumntype{C}{>{         \centering\arraybackslash} p{.245\linewidth}}
\newcolumntype{O}{>{\ttfamily\centering\arraybackslash} p{.200\linewidth}}

\rowcolors{1}{tabhighlight}{white}
\begin{center}
\begin{tabularx}
	{\linewidth}
	{CO|CO}
\toprule[1pt]
	
	\textbf{Vergleich}  & \normalfont \textbf{Zeichen}  &  
	\textbf{Vergleich}  & \normalfont \textbf{Zeichen}
\tabcrlf

	Gleichheit          & ==                   &  Ungleichheit         & != \textrm{oder} <>\\
	Kleiner als         & <                    &  Größer als           & >  \\
	Kleiner oder gleich & <=                   &  Größer oder gleich   & >= \\
	
\bottomrule[1pt]
\end{tabularx}
\end{center}
%
\end{frame}

% =========================================================================== %

\begin{frame}[fragile]
%
\begin{minipage}[t]{.49\linewidth}
\begin{Large}
Logische Verknüpfungen
\vspace{6pt}
\end{Large}
\begin{itemize}
\item \enquote{Verkettung} von Wahrheitswerten
\item \inPy{and}: Beide Aussagen erfüllt
\item \inPy{or}: Mindestens Eine Aussage erfüllt
\item Direkt an \inPy{bool}s (Beispiel rechts)...
\item ... oder in Zusammen gesetzten Ausdrücken\\
	\inPy{good = (T > 20) and (T < 25)}
\item In anderen Sprachen: XOR (exclusive or). Python: Ungleichheit an \inPy{bool}s (\texttt{!=})
\end{itemize}
\end{minipage}
%
\begin{minipage}[t]{.49\linewidth}
\phantom{x}
\begin{codebox}[Code: Verkettung mit \texttt{and}]
\begin{minted}[linenos, fontsize=\scriptsize]{python}
temperature = float(input(
    "Aktuelle Temperatur? "
))
warmEnough = temperature > 20
coldEnough = temperature < 25
pleasantTemperature = warmEnough and \
                      coldEnough
print("'Es ist angenehm warm' ist",
      pleasantTemperature)
\end{minted}
\end{codebox}
\begin{cmdbox}[Ausgabe: Dateneingabe]
\begin{minted}[fontsize=\scriptsize]{text}
Aktuelle Temperatur? 18
'Es ist angenehm warm' ist False
\end{minted}
\end{cmdbox}
\end{minipage}
%
\end{frame}

% =========================================================================== %

\begin{frame}[fragile]{Bedingte Ausführung: \inPy{if}-Blöcke}
%
\begin{minipage}[t]{.49\linewidth}
\begin{itemize}
\item Wenn-Dann-Struktur: Führe Code-Abschnitt nur unter Bedingung aus
\item Bedingung: Wahrheitswert
\item Anweisungen: Code, wie bereits bekannt und wie noch gezeigt. Auch weitere \inPy{if}-Blöcke
\item Einrückungen geben vor, was an Bedingung gebunden
\item Nach dem \inPy{if}-Block: Wieder ohne Einrückung weiter
\end{itemize}
\end{minipage}
%
\begin{minipage}[t]{.49\linewidth}
\phantom{x}
\begin{codebox}[Syntax: \texttt{if} (1)]
\begin{minted}[fontsize=\scriptsize]{python}
if Wahrheitswert :
    Anweisungen
    ...
\end{minted}
\end{codebox}
%
\begin{codebox}[Syntax: \texttt{if} (2)]
\begin{minted}[fontsize=\scriptsize]{python}
if Wahrheitswert :
    Anweisungen
    ...
else :
    Anweisungen
    ...
\end{minted}
\end{codebox}
\end{minipage}
%
\end{frame}

% =========================================================================== %

\begin{frame}{Einrückungen}
%
\begin{itemize}
\item Leerzeichen oder Tabulatoren
\item Nicht gemischt
\item Gleiche Einrückungs-Tiefe über den gesamten Code
\end{itemize}
%
\begin{hintbox}
Empfehlung der Python-Entwickler: Vier Leerzeichen\\
Viele Editoren automatisieren dies mit Tabulator-Taste
\end{hintbox}
%
\end{frame}

% =========================================================================== %

\begin{frame}[fragile]
%
\begin{codebox}[Beispiel]
\begin{minted}[linenos, fontsize=\scriptsize]{python}
temperature = float(input("Bitte geben Sie die aktuelle Temperatur ein: "))

if (temperature > 20) and (temperature < 25) :
    print("Es ist angenehm warm.")
else :
    print("Es ist entweder zu heiß oder zu kalt.")
\end{minted}
\end{codebox}
%
\begin{cmdbox}[Ausgabebeispiel 1]
\begin{minted}[fontsize=\scriptsize]{text}
Aktuelle Temperatur? 23
Es ist angenehm warm.
\end{minted}
\end{cmdbox}
%
\begin{cmdbox}[Ausgabebeispiel 2]
\begin{minted}[fontsize=\scriptsize]{text}
Aktuelle Temperatur? 18
Es ist entweder zu heiß oder zu kalt.
\end{minted}
\end{cmdbox}
%
\end{frame}

% =========================================================================== %

\begin{frame}[fragile]{\inPy{if} mit mehreren Bedingungen}
%
\begin{minipage}[t]{.49\linewidth}
\begin{itemize}
\item Prüft zuerst oberste Bedingung
\item Wenn nicht erfüllt: erstes \inPy{elif}; dann zweites, ...
\item Wenn mehrere Bedingungen erfüllt: nur erster Code ausgeführt
\item Beliebig viele \inPy{elif}-Elemente
\item \inPy{else} weiterhin optional
\end{itemize}
\end{minipage}
%
\begin{minipage}[t]{.49\linewidth}
\phantom{x}
\begin{codebox}[Syntax: \texttt{if} (3)]
\begin{minted}[fontsize=\scriptsize]{python}
if Wahrheitswert :
    Anweisungen
    ...
elif Wahrheitswert :
   Anweisungen
   ...
else :
    Anweisungen
    ...
\end{minted}
\end{codebox}
\end{minipage}
%
\end{frame}

% =========================================================================== %

\begin{frame}[fragile]
%
\begin{codebox}[Beispiel: \texttt{elif}]
\begin{minted}[linenos, fontsize=\scriptsize]{python}
temperature = float(input("Bitte geben Sie die aktuelle Temperatur ein: "))
if temperature < 20 :
    print("Es ist zu kalt.")
elif temperature > 25 :
    print("Es ist zu heiß")
else :
    print("Es ist angenehm warm.")
\end{minted}
\end{codebox}
%
\begin{warnbox}[Fehlerhafter Code: \texttt{elif}, leftupper=6mm]
\begin{minted}[linenos, fontsize=\scriptsize]{python}
i = 23

if i < 25 :
    print("Diese Zahl ist kleiner als 25.")
elif i > 20 :
    print("Diese Zahl ist größer als 20.")
\end{minted}
\end{warnbox}
%
\end{frame}

% =========================================================================== %

\begin{frame}[fragile]
%
\begin{codebox}[Beispiel: Verschachtelter \texttt{if}-Block]
\begin{minted}[linenos, fontsize=\scriptsize]{python}
i = int(input("Bitte geben Sie eine Ganzahl ein: "))

if i % 2 == 0 :
    if   i > 100 :
        print(i, "ist eine große gerade Zahl.")
    elif i >   0 :
        print(i, "ist eine kleine gerade Zahl.")
    else :
        print(i, "ist eine negative gerade Zahl.")
else :
    print(i "ist ungerade.")
\end{minted}
\end{codebox}
%
\end{frame}

% =========================================================================== %

\begin{frame}[fragile]
%
\begin{warnbox}[Redundanz bei \texttt{if} mit logischen Operatoren: \texttt{elif}, leftupper=6mm]
\begin{minted}[linenos, fontsize=\scriptsize]{python}
i = int(input("Bitte geben Sie eine Ganzahl ein: "))

if   i % 2 == 0 and i > 100 :
    print(i, "ist eine große gerade Zahl.")
elif i % 2 == 0 and i <   0 :
    print(i, "ist eine große gerade Zahl.")
elif i % 2 == 0 :
    print(i, "ist eine negative gerade Zahl.")
else :
    print(i "ist ungerade.")
\end{minted}
\end{warnbox}
%
\begin{itemize}
\item Dreifache Prüfung: \texttt{i \% 2}
\item Tippfehler und vergessene Zeilen bei Nachbearbeitung
\item Weniger klar Lesbar: \inPy{else}
\end{itemize}
%
\end{frame}

% =========================================================================== %

\begin{frame}[fragile]
%
\begin{tcbraster}[raster columns=2,
                  raster equal height,
                  nobeforeafter,
                  raster column skip=0.5cm]
	\begin{warnbox}[Gültigkeitsprüfung: Reihe von \texttt{if}s, leftupper=7mm]
	\begin{minted}[linenos, fontsize=\scriptsize]{python}
playerCount = int(input(
    "Bitte Spielerzahl eingeben: "
))
      
if playerCount < 2 :
    print("Ungeeignet für", 
          playerCount, 
          "Spieler."
    )
if playerCount > 5) :
    print("Ungeeignet für", 
          playerCount,
          "Spieler."
    )
	\end{minted}
	\end{warnbox}
%
	\begin{codebox}[Gültigkeitsprüfung: logisches Oder]
	\begin{minted}[linenos, fontsize=\scriptsize]{python}
playerCount = int(input(
    "Bitte Spielerzahl eingeben: "
))
      
if  playerCount < 2 or
    playerCount > 5  :
        print("Ungeeignet für",
              playerCount,
              "Spieler."
        )
	\end{minted}
	\end{codebox}
\end{tcbraster}
%
\end{frame}

% =========================================================================== %

\begin{frame}
%
\begin{hintbox}[Fazit]
Strukturen durchdenken. Redundanz vermeiden. Gegebenenfalls neu aufsetzen.\\

Für größere Projekte schreibe ich fast jede Zeile doppelt: einmal als Konzept, und einmal, um den Code auch über lange Zeit nutzbar zu halten.
\end{hintbox}
%
\end{frame}