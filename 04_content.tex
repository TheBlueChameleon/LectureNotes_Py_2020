% =========================================================================== %

\begin{frame}[t,plain]
\titlepage
\end{frame}

% =========================================================================== %

\begin{frame}{Recap}
%
\begin{columns}[T]
\column{.5\linewidth}
\begin{itemize}
\item Module
	\begin{itemize}
	\item Laden mit \inPy{import Modulname}
	\item Objekte ansprechen mit \inPy{Modulname.Symbol}
	\end{itemize}
\item lists
	\begin{itemize}
	\item Datencontainer
	\item Elemente in [eckigen Klammern, Kommagetrennt]
	\item Zugriff mit [Index in eckigen Klammern]
	\item negative Indices: von hinten herein zählen
	\end{itemize}
\end{itemize}
%
\column{.5\linewidth}
\begin{itemize}
\item Slices
	\begin{itemize}
	\item Index-Variante
	\item Erzeugt neue Liste
	\item \phantom{x} [Start : Stop : Schrittweite]
	\item Elemente können ausgelassen werden
	\end{itemize}
\item Referenzen und Kopieren
	\begin{itemize}
	\item Werzuweisung mit \texttt{=} über Referenzen
	\item Modul Copy
	\item \inPy{copy.deepcopy} für Unterlisten
	\end{itemize}
\item mutable / immutable
\end{itemize}

\end{columns}
%
\begin{center}
	\emph{Noch Fragen?}
\end{center}
%
\end{frame}

% =========================================================================== %

\begin{frame}[fragile]{Kapitel 4}
%
\begin{itemize}
\item \inPy{str}ings
\item \inPy{tuple}s
\item \inPy{set}s und \inPy{frozenset}s
\item \inPy{ranges}
\item \inPy{dict}ionaries
\item Spezielle Funktionen und Operatoren für Container
\end{itemize}
%
\end{frame}

% =========================================================================== %

\begin{frame}[fragile]{Strings}
%
\begin{columns}[T]
\column{.5\linewidth}
\begin{itemize}
\item Bereits bekannt -- Texte in \texttt{''}doppelten Anführungszeichen\texttt{''}
\item Eigentlich: \emph{immutable list of characters}
\item Erlaubt bekannte Techniken:
	\begin{itemize}
	\item lesender Index-Zugriff -- einzelnen Buchstaben herausgreifen
	\item Slicing -- Substring herausnehmen
	\end{itemize}
\item Aber: \emph{Immutable}
	\begin{itemize}
	\item Kann nur als ganzes neu erstellt werden
	\item Einzelne Buchstaben austauschen geht nicht
	\end{itemize}
\end{itemize}
%
\column{.5\linewidth}
\begin{codebox}[Beispiel: Strings]
\begin{minted}[linenos, fontsize=\scriptsize]{python}
text = "These go to eleven"

print(text[0])     # 'T'
print(text[-1])    # 'n'
print(text[:5])    # 'These'

# text[0] = "t"    -- kein Schreibzugriff

text = "It's one louder" 
# Neukonstruktion okay
\end{minted}
\end{codebox}
\end{columns}
%
\end{frame}

% =========================================================================== %

\begin{frame}[fragile]{\inPy{tuple}s}
%
\begin{columns}[T]
\column{.5\linewidth}
\begin{itemize}
\item Wie Lists: Datencontainer
\item Aber: \emph{immutable}
\item In (runden Klammern) eingefasst
\item Kein \texttt{append}, \texttt{delete}, \texttt{sort}, ...
\item Addition konstruiert neuen \inPy{tuple}
\item Häufig Rückgabetyp von Funktionen
	\begin{itemize}
	\item Beispiel \inPy{divmod} Quotient und Rest als \inPy{tuple}
	\item \inPy{divmod(20, 3)} \thus ~ \inPy{(6, 2)}
	\end{itemize}

\end{itemize}
%
\column{.5\linewidth}
\begin{codebox}[Beispiel: \texttt{tuple}s]
\begin{minted}[linenos, fontsize=\scriptsize]{python}
t = (1, 2, 3)
print(t)        #  (1, 2, 3)
print(t[0])     #  1
# t[0] = -1     -- kein Schreibzugriff
t = (-1, -2, -3)

t = tuple([4, 5, 6])
# Umwandlung list zu tuple

t = tuple("text")
# Umwandlung String zu tuple.
# ('t', 'e', 'x', 't')
\end{minted}
\end{codebox}
\end{columns}
%
\end{frame}

% =========================================================================== %

\begin{frame}[fragile]{\inPy{set}s und \inPy{frozenset}s}
%
\begin{columns}[T]
\column{.5\linewidth}
\begin{itemize}
\item \inPy{set}s
	\begin{itemize}
	\item Datencontainer für \emph{eindeutige Werte}
	\item \emph{mutable}
	\item In \{geschweifte Klammern\} eingefasst
	\item Wichtige Methoden: \texttt{add}, \texttt{clear}, \texttt{difference}, \texttt{intersection}, \texttt{union}
	\item Addition nicht definiert für \inPy{set}s
	\end{itemize}
\item \inPy{frozenset}s
	\begin{itemize}
	\item Wie \inPy{set}s, aber \emph{immutable}
	\item kein eigenes Symbol, sondern über Constructor: \inPy{variable = frozenset(Liste)}
	\end{itemize}
\end{itemize}
%
\column{.5\linewidth}
\begin{codebox}[Beispiel: \texttt{set}s]
\begin{minted}[linenos, fontsize=\scriptsize]{python}
s = {1, 2, 3, 2}
print(s)        #  (1, 2, 3)
print(t[0])     #  1
# s[0] = -1     
# kein Schreibzugriff auf Elemente

s.add(4)        # aber Änderung möglich
s = s.intersection({2, 4, 6})
print(s)        # {2, 4}

s = set("aardvark")
print(s)        #{'r', 'v', 'k', 'd', 'a'}
# Sortierung geht ggf. verloren.
\end{minted}
\end{codebox}
\end{columns}
%
\end{frame}

% =========================================================================== %

\begin{frame}[fragile]{\inPy{range}s}
%
\begin{columns}[T]
\column{.5\linewidth}
\begin{itemize}
\item Platzsparender Container für Folgen von Ganzzahlen
\item \inPy{Start, Stop, Schrittweite}
\item Speichert tatsächlich nur die 3 Werte, generiert Zahlen \enquote{on demand}
\item \texttt{Start} eingeschlossen, \texttt{Stop} ausgeschlossen
\item \texttt{Schrittweite} kann ausgelassen werden (dann automatisch 1)
\item \texttt{Schrittweite} kann auch negativ sein
\item \texttt{Start}, \texttt{Stop}, \texttt{Schrittweite}: \emph{nur Ganzzahlen}
\end{itemize}
%
\column{.5\linewidth}
\begin{codebox}[Beispiel: \texttt{range}s]
\begin{minted}[linenos, fontsize=\scriptsize]{python}
r = range(1, 10, 2)
print(r)         # range(1, 10, 2)
print(list(r))   # [1, 3, 5, 7, 9]

r = range(1, 5)
print(list(r))   # [1, 2, 3, 4]

print(list(range(5, 0, -1)))
# [5, 4, 3, 2, 1]

\end{minted}
\end{codebox}
\end{columns}
%
\end{frame}

% =========================================================================== %

\begin{frame}[fragile]{\inPy{dict}ionaries}
%
\begin{columns}[T]
\column{.5\linewidth}
\begin{itemize}
\item Liste mit \emph{Schlüsseln} statt Indices
\item Schlüssel: Beliebiger Wert, \zB ein String
\item \{Geschweifte Klammern\}, Schlüssel-Wert-Paare durch Doppelpunkt verbunden
\item Wichtige Methoden:
	\begin{itemize}
	\item \texttt{keys} -- gibt eine Liste von Schlüsseln aus
	\item \texttt{values} -- gibt eine Liste von Werten aus
	\item \texttt{items} -- gibt eine Liste von Tupeln aus Schlüssel und Wert aus
	\end{itemize}
\end{itemize}
%
\column{.5\linewidth}
\begin{codebox}[Beispiel: \texttt{range}s]
\begin{minted}[linenos, fontsize=\scriptsize]{python}
d = {'north' : 12, 'east' : 9,
     'south' :  6, 'west' : 3}

print(d['north'])      # 12
print(list(d.keys()))
# ['north', 'east', 'south', 'west']

print(list(d.items())[0:2])
# [('north', 12), ('east', 9)]

d['north'] = 0
# auch Schreibzugriff möglich
\end{minted}
\end{codebox}
\end{columns}
%
\end{frame}

% =========================================================================== %

\begin{frame}[fragile]{Spezielle Funktionen und Operatoren für Container (I)}
%
\begin{itemize}
\item \inPy{in}: Boolean, ob Wert in Container ist.
	\begin{itemize}
	\item \inPy{1 in [1, 2, 3]} gibt \inPy{True}
	\item \inPy{[1] in [1, 2, 3]} gibt \inPy{False}
	\item \inPy{1 in {1 : 'a', 2 : 'b'}} gibt \inPy{True}
	\item \inPy{'a' in {1 : 'a', 2 : 'b'}} gibt \inPy{False}
	\item \inPy{'a' in {1 : 'a', 2 : 'b'}.values()} gibt \inPy{True}
	\end{itemize}
\item \inPy{len} -- Anzahl Elemente in einem Container
\item \inPy{reversed} und \inPy{sorted} -- \emph{Kopien} mit umgedrehter Reihenfolge bzw. Sortierung
\end{itemize}
%
\end{frame}

% =========================================================================== %

\begin{frame}[fragile]{Spezielle Funktionen und Operatoren für Container (II)}
%
\begin{itemize}
\item Vergleichsoperatoren (\texttt{==}, \texttt{<}, \texttt{>})
	\begin{itemize}
	\item Elementweiser Vergleich, Entscheidung beim ersten nicht-gleichen Wert
	\item Wie bei Strings: \inPy{"aardvark" < "antilope"} (Entscheidung beim zweiten Zeichen, da erstes Zeichen gleich)
	\item \inPy{[1, 5, 7] < [1, 5, 9, -3]} aus selbem Grund.
	\end{itemize}
\item \inPy{zip} -- mehrere Listen zu Tupeln zusammenschnüren\\
	\begin{codebox}[Beispiel: \texttt{zip}]
	\begin{minted}[linenos, fontsize=\scriptsize]{python}
A = ["Gryffindor", "Ravenclaw", "Hufflepuff", "Slytherin"]
B = ["red", "blue", "yellow", "green"]
print(list(zip(A, B)))
	\end{minted}
	\end{codebox}
	\begin{cmdbox}[Ausgabe: \texttt{zip}]
	\begin{minted}[fontsize=\scriptsize]{text}
[("Gryffindor", "red"), ("Ravenclaw", "blue"),
 ("Hufflepuff", "yellow"), ("Slytherin", "green")]
	\end{minted}
	\end{cmdbox}
\end{itemize}
%
\end{frame}