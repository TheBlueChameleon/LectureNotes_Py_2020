% =========================================================================== %

\begin{frame}[t,plain]
\titlepage
\end{frame}

% =========================================================================== %

\begin{frame}{Recap}
%
\begin{columns}[T]
\column{.5\linewidth}
\begin{itemize}
\item Module
	\begin{itemize}
	\item Laden mit \inPy{import Modulname}
	\item Objekte ansprechen mit \inPy{Modulname.Symbol}
	\end{itemize}
\item lists
	\begin{itemize}
	\item Datencontainer
	\item Elemente in [eckigen Klammern, Kommagetrennt]
	\item Zugriff mit [Index in eckigen Klammern]
	\item negative Indices: von hinten herein zählen
	\end{itemize}
\end{itemize}
%
\column{.5\linewidth}
\begin{itemize}
\item Slices
	\begin{itemize}
	\item Index-Variante
	\item Erzeugt neue Liste
	\item \phantom{x} [Start : Stop : Schrittweite]
	\item Elemente können ausgelassen werden
	\end{itemize}
\item Referenzen und Kopieren
	\begin{itemize}
	\item Werzuweisung mit \texttt{=} über Referenzen
	\item Modul Copy
	\item \inPy{copy.deepcopy} für Unterlisten
	\end{itemize}
\item mutable / immutable
\end{itemize}

\end{columns}
%
\begin{center}
	\emph{Noch Fragen?}
\end{center}
%
\end{frame}

% =========================================================================== %

\begin{frame}[fragile]{Kapitel 4}
%
\begin{itemize}
\item \inPy{str}ings
\item \inPy{tuple}s
\item \inPy{set}s und \inPy{frozenset}s
\item \inPy{ranges}
\item \inPy{dict}ionaries
\item Spezielle Funktionen und Operatoren für Container
\end{itemize}
%
\end{frame}

% =========================================================================== %

\begin{frame}[fragile]{Strings}
%
\begin{columns}[T]
\column{.5\linewidth}
\begin{itemize}
\item Bereits bekannt -- Texte in \texttt{''}doppelten Anführungszeichen\texttt{''}
\item Eigentlich: \emph{immutable list of characters}
\item Erlaubt bekannte Techniken:
	\begin{itemize}
	\item lesender Index-Zugriff -- einzelnen Buchstaben herausgreifen
	\item Slicing -- Substring herausnehmen
	\end{itemize}
\item Aber: \emph{Immutable}
	\begin{itemize}
	\item Kann nur als ganzes neu erstellt werden
	\item Einzelne Buchstaben austauschen geht nicht
	\end{itemize}
\end{itemize}
%
\column{.5\linewidth}
\begin{codebox}[Beispiel: Strings]
\begin{minted}[linenos, fontsize=\scriptsize]{python}
text = "These go to eleven"

print(text[0])     # 'T'
print(text[-1])    # 'n'
print(text[:5])    # 'These'

# text[0] = "t"    -- kein Schreibzugriff

text = "It's one louder" 
# Neukonstruktion okay
\end{minted}
\end{codebox}
\end{columns}
%
\end{frame}

% =========================================================================== %

\begin{frame}[fragile]{\inPy{tuple}s}
%
\begin{columns}[T]
\column{.5\linewidth}
\begin{itemize}
\item Wie Lists: Datencontainer
\item Aber: \emph{immutable}
\item In (runden Klammern) eingefasst
\item Kein \texttt{append}, \texttt{delete}, \texttt{sort}, ...
\item Addition konstruiert neuen \inPy{tuple}
\item Häufig Rückgabetyp von Funktionen
	\begin{itemize}
	\item Beispiel \inPy{divmod} Quotient und Rest als \inPy{tuple}
	\item \inPy{divmod(20, 3)} \thus ~ \inPy{(6, 2)}
	\end{itemize}
\item Entpacken: \inPy{x, y = (1, 2)}
\end{itemize}
%
\column{.5\linewidth}
\begin{codebox}[Beispiel: \texttt{tuple}s]
\begin{minted}[linenos, fontsize=\scriptsize]{python}
t = (1, 2, 3)
print(t)        #  (1, 2, 3)
print(t[0])     #  1
# t[0] = -1     -- kein Schreibzugriff
t = (-1, -2, -3)

t = tuple([4, 5, 6])
# Umwandlung list zu tuple

t = tuple("text")
# Umwandlung String zu tuple.
# ('t', 'e', 'x', 't')
\end{minted}
\end{codebox}
\end{columns}
%
\end{frame}

% =========================================================================== %

\begin{frame}[fragile]{\inPy{set}s und \inPy{frozenset}s}
%
\begin{columns}[T]
\column{.5\linewidth}
\begin{itemize}
\item \inPy{set}s
	\begin{itemize}
	\item Datencontainer für \emph{eindeutige Werte}
	\item \emph{mutable}
	\item In \{geschweifte Klammern\} eingefasst
	\item Wichtige Methoden: \texttt{add}, \texttt{clear}, \texttt{difference}, \texttt{intersection}, \texttt{union}
	\item Addition nicht definiert für \inPy{set}s
	\end{itemize}
\item \inPy{frozenset}s
	\begin{itemize}
	\item Wie \inPy{set}s, aber \emph{immutable}
	\item kein eigenes Symbol, sondern über Constructor: \inPy{variable = frozenset(Liste)}
	\end{itemize}
\end{itemize}
%
\column{.5\linewidth}
\begin{codebox}[Beispiel: \texttt{set}s]
\begin{minted}[linenos, fontsize=\scriptsize]{python}
s = {1, 2, 3, 2}
print(s)        #  {1, 2, 3}
print(s[0])     #  1
# s[0] = -1     
# kein Schreibzugriff auf Elemente

s.add(4)        # aber Änderung möglich
s = s.intersection({2, 4, 6})
print(s)        # {2, 4}

s = set("aardvark")
print(s)        #{'r', 'v', 'k', 'd', 'a'}
# Sortierung geht ggf. verloren.
\end{minted}
\end{codebox}
\end{columns}
%
\end{frame}

% =========================================================================== %

\begin{frame}[fragile]{\inPy{range}s}
%
\begin{columns}[T]
\column{.5\linewidth}
\begin{itemize}
\item Platzsparender Container für Folgen von Ganzzahlen
\item \inPy{Start, Stop, Schrittweite}
\item Speichert tatsächlich nur die 3 Werte, generiert Zahlen \enquote{on demand}
\item \texttt{Start} eingeschlossen, \texttt{Stop} ausgeschlossen
\item \texttt{Schrittweite} kann ausgelassen werden (dann automatisch 1)
\item \texttt{Schrittweite} kann auch negativ sein
\item \texttt{Start}, \texttt{Stop}, \texttt{Schrittweite}: \emph{nur Ganzzahlen}
\end{itemize}
%
\column{.5\linewidth}
\begin{codebox}[Beispiel: \texttt{range}s]
\begin{minted}[linenos, fontsize=\scriptsize]{python}
r = range(1, 10, 2)
print(r)         # range(1, 10, 2)
print(list(r))   # [1, 3, 5, 7, 9]

r = range(1, 5)
print(list(r))   # [1, 2, 3, 4]

print(list(range(5, 0, -1)))
# [5, 4, 3, 2, 1]

\end{minted}
\end{codebox}
\end{columns}
%
\end{frame}

% =========================================================================== %

\begin{frame}[fragile]{\inPy{dict}ionaries}
%
\begin{columns}[T]
\column{.5\linewidth}
\begin{itemize}
\item Liste mit \emph{Schlüsseln} statt Indices
\item Schlüssel: Beliebiger Wert, \zB ein String
\item \{Geschweifte Klammern\}, Schlüssel-Wert-Paare durch Doppelpunkt verbunden
\item Wichtige Methoden:
	\begin{itemize}
	\item \texttt{keys} -- gibt eine Liste von Schlüsseln aus
	\item \texttt{values} -- gibt eine Liste von Werten aus
	\item \texttt{items} -- gibt eine Liste von Tupeln aus Schlüssel und Wert aus
	\end{itemize}
\end{itemize}
%
\column{.5\linewidth}
\begin{codebox}[Beispiel: \texttt{range}s]
\begin{minted}[linenos, fontsize=\scriptsize]{python}
d = {'north' : 12, 'east' : 9,
     'south' :  6, 'west' : 3}

print(d['north'])      # 12
print(list(d.keys()))
# ['north', 'east', 'south', 'west']

print(list(d.items())[0:2])
# [('north', 12), ('east', 9)]

d['north'] = 0
# auch Schreibzugriff möglich
\end{minted}
\end{codebox}
\end{columns}
%
\end{frame}

% =========================================================================== %

\begin{frame}[fragile]{Spezielle Funktionen und Operatoren für Container (I)}
%
\begin{itemize}
\item \inPy{in}: Boolean, ob Wert in Container ist.
	\begin{itemize}
	\item \inPy{1 in [1, 2, 3]} gibt \inPy{True}
	\item \inPy{[1] in [1, 2, 3]} gibt \inPy{False}
	\item \inPy{1 in {1 : 'a', 2 : 'b'}} gibt \inPy{True}
	\item \inPy{'a' in {1 : 'a', 2 : 'b'}} gibt \inPy{False}
	\item \inPy{'a' in {1 : 'a', 2 : 'b'}.values()} gibt \inPy{True}
	\end{itemize}
\item \inPy{len} -- Anzahl Elemente in einem Container
\item \inPy{reversed} und \inPy{sorted} -- \emph{Kopien} mit umgedrehter Reihenfolge bzw. Sortierung
\end{itemize}
%
\end{frame}

% =========================================================================== %

\begin{frame}[fragile]{Spezielle Funktionen und Operatoren für Container (II)}
%
\begin{itemize}
\item Vergleichsoperatoren (\texttt{==}, \texttt{<}, \texttt{>})
	\begin{itemize}
	\item Elementweiser Vergleich, Entscheidung beim ersten nicht-gleichen Wert
	\item Wie bei Strings: \inPy{"aardvark" < "antilope"} (Entscheidung beim zweiten Zeichen, da erstes Zeichen gleich)
	\item \inPy{[1, 5, 7] < [1, 5, 9, -3]} aus selbem Grund.
	\end{itemize}
\item \inPy{zip} -- mehrere Listen zu Tupeln zusammenschnüren\\
	\begin{codebox}[Beispiel: \texttt{zip}]
	\begin{minted}[linenos, fontsize=\scriptsize]{python}
A = ["Gryffindor", "Ravenclaw", "Hufflepuff", "Slytherin"]
B = ["red", "blue", "yellow", "green"]
print(list(zip(A, B)))
	\end{minted}
	\end{codebox}
	\begin{cmdbox}[Ausgabe: \texttt{zip}]
	\begin{minted}[fontsize=\scriptsize]{text}
[("Gryffindor", "red"), ("Ravenclaw", "blue"),
 ("Hufflepuff", "yellow"), ("Slytherin", "green")]
	\end{minted}
	\end{cmdbox}
\end{itemize}
%
\end{frame}

% =========================================================================== %

\begin{frame}{Kapitel 5}
%
\begin{itemize}
\item \inPy{for}-Schleifen
\item Entpacken von \inPy{tuple}s
\item List Comprehension
\end{itemize}
%
\end{frame}

% =========================================================================== %

\begin{frame}[fragile]{\inPy{for}-Schleifen}
%
\begin{columns}[T]
\column{.5\linewidth}
\begin{itemize}
\item Idee: Führe [Aktionen] \emph{für jedes Element} in einem Container durch
\item Container: \inPy{list}s, \inPy{tuple}s, \inPy{range}s, \inPy{dict}s, ...
\item Aktion: Beliebiger Python-Code wie schon gesehen und noch zu zeigen
\item Mehrzeilige Aktionen \Thus Einrückungsebene
\item Verschachtelung möglich
\item Vergleiche Form von \inPy{if}-Statements
\end{itemize}
%
\column{.5\linewidth}
\begin{codebox}[Syntax: \texttt{for}-Schleifen]
\begin{minted}[fontsize=\scriptsize]{python}
for Element in Container :
    Aktionen

weiterer Code (unabhängig von Schleife)
\end{minted}
\end{codebox}
%
\begin{itemize}
\item \texttt{Element} ist also eine neue Variable
\item Wird nacheinander mit den Elementen des Containers \enquote{befüllt}
\end{itemize}
\end{columns}
%
\end{frame}

% =========================================================================== %

\begin{frame}[fragile]
%
\begin{codebox}[Beispiel: \texttt{for}-Schleifen mit \texttt{list}s]
\begin{minted}[fontsize=\scriptsize, linenos]{python}
tasklist = ["write the script", "drink some coffee", "drink some more coffee"]

print("Your tasks today:")
for task in tasklist :
    print("*", task)

print("Do it now!")
\end{minted}
\end{codebox}
%
\begin{cmdbox}[Ausgabe: \texttt{for}-Schleifen mit \texttt{list}s]
\begin{minted}[fontsize=\scriptsize]{text}
Your tasks today:
* write the script
* drink some coffee
* drink some more coffee"
Do it now!
\end{minted}
\end{cmdbox}
%
\end{frame}

% =========================================================================== %

\begin{frame}[fragile]
%
\begin{codebox}[Beispiel: \texttt{for}-Schleifen mit \texttt{range}s -- Zählschleifen]
\begin{minted}[fontsize=\scriptsize, linenos]{python}
print("the first 10 square numbers are:")
for i in range(10) :
    print(f"{i}² = {i**2}")
\end{minted}
\end{codebox}
%
\begin{cmdbox}[Ausgabe: \texttt{for}-Schleifen mit \texttt{range}s -- Zählschleifen]
\begin{minted}[fontsize=\scriptsize]{text}
the first 10 square numbers are:
0² = 0
1² = 1
2² = 4
3² = 9
4² = 16
5² = 25
6² = 36
7² = 49
8² = 64
9² = 81
\end{minted}
\end{cmdbox}
%
\end{frame}

% =========================================================================== %

\begin{frame}[fragile]
%
\begin{codebox}[Beispiel: \texttt{for}-Schleifen mit Entpacken von \texttt{tuple}s]
\begin{minted}[fontsize=\scriptsize, linenos]{python}
books = [("Frank Herbert", "Dune"),
         ("Douglas Adams", "The Hitchhikers Guide To The Galaxy"),
         ("Randall Munroe", "What If"),
         ("Isaac Asimov", "Foundation"),
         ("Willy Russell", "Educating Rita"),
         ("Moving Pictures", "Terry Pratchett")]
print("You should definitively read:")
for author, title in books :
    print("*", title, "by", author)
\end{minted}
\end{codebox}
%
\begin{cmdbox}[Ausgabe: \texttt{for}-Schleifen mit Entpacken von \texttt{tuple}s]
\begin{minted}[fontsize=\scriptsize]{text}
You should definitively read:
* Dune by Frank Herbert
* The Hitchhikers Guide To The Galaxy by Douglas Adams
* What If by Randall Munroe
* Foundation by Isaac Asimov
* Educating Rita by Willy Russell
* Moving Pictures by Terry Pratchett
\end{minted}
\end{cmdbox}
%
\end{frame}

% =========================================================================== %

\begin{frame}[fragile]
%
\begin{codebox}[Beispiel: \texttt{for}-Schleifen ohne Entpacken von \texttt{tuple}s]
\begin{minted}[fontsize=\scriptsize, linenos]{python}
import math
vectors = [(1, 1), (4, 7), (-1, 2)]
for v in vectors :
    print("vector", v, "has length", math.hypot(v[0], v[1]))
\end{minted}
\end{codebox}
%
\begin{cmdbox}[Ausgabe: \texttt{for}-Schleifen ohne Entpacken von \texttt{tuple}s]
\begin{minted}[fontsize=\scriptsize]{text}
vector (1, 1) has length 1.4142135623730951
vector (4, 7) has length 8.06225774829855
vector (-1, 2) has length 2.23606797749979
\end{minted}
\end{cmdbox}
%
\end{frame}

% =========================================================================== %

\begin{frame}[fragile]
%
\begin{codebox}[Beispiel: \texttt{for}-Schleifen mit \texttt{enumerate}]
\begin{minted}[fontsize=\scriptsize, linenos]{python}
tasklist = ["write the script", "drink some coffee", "drink more coffee"]
print("Your tasks today:")
for i, task in enumerate(tasklist) :
    print(f"{i + 1}. {task}")
\end{minted}
\end{codebox}
%
\begin{cmdbox}[Ausgabe: \texttt{for}-Schleifen mit \texttt{enumerate}]
\begin{minted}[fontsize=\scriptsize]{text}
Your tasks today:
1. write the script
2. drink some coffee
3. drink more coffee
\end{minted}
\end{cmdbox}
%
\begin{hintbox}[\texttt{enumerate}]
Der Befehl \inPy{enumerate} generiert \inPy{tuple}s aus Index und Container-Elementen.
\end{hintbox}
%
\end{frame}

% =========================================================================== %

\begin{frame}[fragile]
%
\begin{codebox}[Beispiel: \texttt{for}-Schleifen mit \texttt{zip}]
\begin{minted}[fontsize=\scriptsize, linenos]{python}
data1 = [1.7, 2.2, -4.1]
data2 = [1.8, 2.0, -3.8]
print("Difference in datasets:")
for i, t in enumerate(zip(data1, data2)) :
    print(f"Datapoint {i}: {t} differs by : {t[0] - t[1]}")
\end{minted}
\end{codebox}
%
\begin{cmdbox}[Ausgabe: \texttt{for}-Schleifen mit \texttt{zip}]
\begin{minted}[fontsize=\scriptsize]{text}
Datapoint 0: (1.7, 1.8) differs by : -0.10000000000000009
Datapoint 1: (2.2, 2.0) differs by : 0.20000000000000018
Datapoint 2: (-4.1, -3.8) differs by : -0.2999999999999998
\end{minted}
\end{cmdbox}
%
\begin{hintbox}[\texttt{zip}]
Der Befehl \inPy{zip} generiert \inPy{tuple}s aus Container-Elementen mit gleichem Index.
\end{hintbox}
%
\end{frame}

% =========================================================================== %

\begin{frame}[fragile]
%
\begin{codebox}[Beispiel: \texttt{for}-Schleifen mit \texttt{dict}s]
\begin{minted}[fontsize=\scriptsize, linenos]{python}
houseStark = {"Sigil" : "A grey direwolf on a white field",
              "Words" : "Winter Is Coming",
              "Seat"  : "Winterfell"}

print("Summary of House Stark:")
for key, value in houseStark.items() :
    print(f"{key:5}: {value}")
\end{minted}
\end{codebox}
%
\begin{cmdbox}[Ausgabe: \texttt{for}-Schleifen mit \texttt{dict}s]
\begin{minted}[fontsize=\scriptsize]{text}
Summary of House Stark:
Sigil: A grey direwolf on a white field
Words: Winter Is Coming
Seat : Winterfell
\end{minted}
\end{cmdbox}
%
\end{frame}

% =========================================================================== %

\begin{frame}[fragile]{List Comprehension}
%
Idee: Schnelles/Bequemes Erstellen von Containern aus \inPy{for}-Schleife
\begin{codebox}[Syntax: List Comprehension]
\begin{minted}[fontsize=\scriptsize]{python}
listVariable = [            Ausdruck for Element in Container if Bedingung]
setVariable  = {            Ausdruck for Element in Container if Bedingung}
dictVariable = {Schlüssel : Ausdruck for Element in Container if Bedingung}
\end{minted}
\end{codebox}
%
\begin{itemize}
\item \texttt{Ausdruck} : Beliebige \enquote{Rechnung}, die ein Listenelement beschreibt
\item \texttt{Element}  : Hilfsvariable, die die Container-Elemente aufnimmt (wie bei \inPy{for})
\item \texttt{Container}: Eine \inPy{list}, \inPy{range}, ...
\item \texttt{Bedingung}: Ausdruck, der angibt, welche Elemente aufgenommen werden sollen
\item \inPy{if Bedingung} ist optional
\end{itemize}
%
\end{frame}

% =========================================================================== %

\begin{frame}[fragile]
%
\begin{codebox}[Beispiel: List Comprehension]
\begin{minted}[fontsize=\scriptsize, linenos]{python}
import math

N = 3
eValues = [math.exp(k) for k in range(N)]

print(eValues)
\end{minted}
\end{codebox}
%
\begin{cmdbox}[Ausgabe: List Comprehension]
\begin{minted}[fontsize=\scriptsize]{text}
[1.0, 2.718281828459045, 7.38905609893065]
\end{minted}
\end{cmdbox}
%
\end{frame}

% =========================================================================== %

\begin{frame}[fragile]
%
\begin{codebox}[Beispiel: Telefon-Matrix mit List Comprehension]
\begin{minted}[fontsize=\scriptsize, linenos]{python}
telephone = [[3 * row + column + 1 for column in range(3)]
             for row in range(3)
            ]

for line in telephone :
    print(line)

print(telephone)
\end{minted}
\end{codebox}
%
\begin{cmdbox}[Ausgabe: Telefon-Matrix mit List Comprehension]
\begin{minted}[fontsize=\scriptsize]{text}
[1, 2, 3]
[4, 5, 6]
[7, 8, 9]
[[1, 2, 3], [4, 5, 6], [7, 8, 9]]
\end{minted}
\end{cmdbox}
%
\end{frame}

% =========================================================================== %

\begin{frame}[fragile]
%
\begin{codebox}[Beispiel: Primzahlen mit List Comprehension]
\begin{minted}[fontsize=\scriptsize, linenos]{python}
N = 100
primeNumbers = [
    i for i in range (2, N)          # übernehme alle Zahlen i zwischen 2 und N ...
    if len (                         # ... für die die die Anzahl der Teiler ...
        [j for j in range (1, i+1)   # ... (d.h. die Zahlen zwischen 1 und i ...
        if i % j == 0]               # ... die i ganzzahlig teilen) ...
    ) == 2                           # ... gleich 2 ist.
]

print(primeNumbers)

\end{minted}
\end{codebox}
%
\begin{cmdbox}[Ausgabe: Primzahlen mit List Comprehension]
\begin{minted}[fontsize=\scriptsize]{text}
[2, 3, 5, 7, 11, 13, 17, 19, 23, 29, 31, 37, 41, 43, 47, 53, 59, 61, 67, 71,
73, 79, 83, 89, 97]
\end{minted}
\end{cmdbox}
%
\end{frame}