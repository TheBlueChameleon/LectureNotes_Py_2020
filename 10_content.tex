% =========================================================================== %

\begin{frame}[t,plain]
\titlepage
\end{frame}

% =========================================================================== %

\begin{frame}{Recap}
%
\begin{columns}[T]
\column{.5\linewidth}
\begin{itemize}
\item Dateizugriff über Handles
	\begin{itemize}
	\item \inPy{handle = open(Dateiname, Modus)}
	\item Vorgefertigte Objekte
	\item Dateimodus: Lesen/Schreiben/Anhängen, Binär/Textmodus
	\item Sollte nach letztem Zugriff geschlossen werden
	\end{itemize}
\item In Dateien Schreiben
	\begin{itemize}
	\item mit \texttt{handle.write}
	\item Argument: String im Textmodus, \inPy{bytes}-Objekt im Binärmodus
	\end{itemize}
\end{itemize}
%
\column{.5\linewidth}
\begin{itemize}
\item Aus Dateien Lesen
	\begin{itemize}
	\item \texttt{handle.read} -- Einlesen als String oder \inPy{bytes}-Objekt
	\item Ganze Datei oder Maximallänge
	\item \texttt{readline} und \texttt{readlines} -- ganze Zeile(n)
	\end{itemize}
\item Blockstrukturen
	\begin{itemize}
	\item \inPy{with open(...) as handle} automatisches Schließen
	\item realisiert über Dunders \inPy{__enter__} und \inPy{__exit__}
	\item \inPy{for line in handle}
	\end{itemize}
\end{itemize}
\end{columns}
%
\begin{center}
	\emph{Noch Fragen?}
\end{center}
%
\end{frame}

% =========================================================================== %

\begin{frame}{Recap}
%
\begin{columns}[T]
\column{.5\linewidth}
\begin{itemize}
\item Cursorposition Manipulieren
	\begin{itemize}
	\item \texttt{handle.tell} -- Aktuelle Position bestimmen
	\item \texttt{handle.seek(Offset, Startpunkt)} -- Cursor verschieben
	\end{itemize}
\item Pickle
	\begin{itemize}
	\item Objekte Archivieren
	\item Funktioniert mit allem
	\item Kann Schadcode enthalten
	\item \texttt{pickle.dump} und \texttt{pickle.load}
	\end{itemize}
\end{itemize}
%
\column{.5\linewidth}
\begin{itemize}
\item JSON
	\begin{itemize}
	\item Austausch mit anderen Programmen
	\item Funktioniert nur mit bestimmten Datentypen
	\item Menschenlesbares Format
	\item \texttt{json.dump} und \texttt{json.load}
	\item Meist über \inPy{dict}s
	\end{itemize}
\item CSV
	\begin{itemize}
	\item Information in Spalten
	\item Objektklasse \texttt{DictReader} von \texttt{csv.reader}
	\item Iterable -- kann mit \inPy{for} benutzt werden
	\item Zeilen an Trennzeichen (\zB \texttt{','}) in Listen aufspalten
	\end{itemize}
\end{itemize}

\end{columns}
%
\begin{center}
	\emph{Noch Fragen?}
\end{center}
%
\end{frame}

% =========================================================================== %

\begin{frame}[fragile]{Aus den Übungen}
%
\begin{tcbraster}[raster columns=2,
                  raster equal height,
                  nobeforeafter,
                  raster column skip=0.5cm]
\begin{codebox}[Beispiel: Title goes here]
\begin{minted}[fontsize=\scriptsize, linenos]{python}
foo
\end{minted}
\end{codebox}
%
\begin{codebox}[Beispiel: Title goes here]
\begin{minted}[fontsize=\scriptsize, linenos]{python}
bar
\end{minted}
\end{codebox}
\end{tcbraster}
%
\end{frame}

% =========================================================================== %

\begin{frame}[fragile]{Kapitel 8}
%
\begin{itemize}
\item Darstellung in Binär und als Text
\item Handles
\item Lesen und Schreiben
\item Blockstrukturen 
\item Dateicursor
\item Pickle und JSON
\end{itemize}
%
\end{frame}

% =========================================================================== %

\begin{frame}[fragile]{Darstellung in Binär und als Text}
%
\begin{columns}[T]
\column{.5\linewidth}
\begin{itemize}
\item Recap: Mensch benutzt Tausende Schriftzeichen, Computer nur zwei
\item Alle Information für Computer nur Binärzahlen
\item Regeln zur Interpretation für Interfaces zum Menschen
\item[\Thus] Alles ist Zahl
\end{itemize}
%
\column{.5\linewidth}
\begin{itemize}
\item Information, die wie Zahlen aussehen, können intern ganz anders abgelegt sein
\item Beispiel: \emph{Text} \inPy{"42"} ist nicht gleich Zahl \inPy{42}
	\begin{itemize}
	\item Text: Zwei Zeichen: \inPy{"4"}, mit ASCII-Code 52 und \inPy{"2"} mit ASCII-Code 50\\
		Bitmuster \texttt{00110110 00110100}
	\item Zahl: Direkte Umrechnung zur Binärzahl\\
		Bitmuster \texttt{00101010}
	\end{itemize}
\end{itemize}
\end{columns}
%
\begin{center}
	\begin{large}
	\Thus \emph{Als ProgrammiererInnen müssen wir im Auge behalten, welche Art von Daten wir interpretieren, und diesen Kontext dem Computer mitgeben!}
	\end{large}
\end{center}
%
\end{frame}

% =========================================================================== %

\begin{frame}[fragile]
%
\begin{columns}[T]
\column{.5\linewidth}
\begin{Large}
	{Dateien Öffnen -- Handles}
	\vspace{6pt}
\end{Large}
\begin{itemize}
\item Python macht die meiste Interpretation im Hintergrund durch zugeordnete Datentypen (\inPy{int}, \inPy{float}, \inPy{dict}, \ldots)
\item Datei-Inhalte haben nur \enquote{rohe Information}, ohne Datentyp.
\item[\Thus] Erst mal: Unsere Aufgabe
\item Python: Dateimodus -- wie Zuweisung eines Datentyps
\item Vorstellung: \enquote{Griff an Datei anbringen}, über den die Datei \enquote{angefasst} werden kann.
\end{itemize}
%
\column{.5\linewidth}
\begin{codebox}[Syntax: Datei Öffnen]
\begin{minted}[fontsize=\scriptsize]{python}
handle = open(Dateiname, Modus)
\end{minted}
\end{codebox}
%
\begin{itemize}
\item \texttt{Dateiname} -- relativer oder absoluter Dateiname
	\begin{itemize}
	\item \inPy{"myFile.txt"} -- wird im aktuellen Arbeitsverzeichnis geöffnet
	\item \inPy{"C:\myFiles\myFile.txt"} -- absolute Angabe
	\item \inPy{"..\myFile.txt"} -- wird im Überordner des aktuellen Arbeitsverzeichnisses geöffnet
	\end{itemize}
\item Linux und Mac: Trennzeichen ist ein \emph{forward slash} (/), nicht de Backslash (\textbackslash).
\end{itemize}
\end{columns}
%
\end{frame}

% =========================================================================== %

\begin{frame}[fragile]{Dateimodi}
%
\begin{center}
\scriptsize
\rowcolors{1}{white}{tabhighlight}
\begin{tabular}{c|cp{.6\linewidth}}
	\toprule
	\textbf{String}	& \textbf{Bedeutung}				& \textbf{Kommentar} \tabcrlf
	\texttt{r}				& Lesen -- Textmodus				& Dateicursor am Anfang der Datei.\newline Neuen Dateien werden \emph{nicht} angelegt.\\
	\texttt{rb}			& Lesen -- Binärmodus			& Dateicursor am Anfang der Datei.\newline Neuen Dateien werden \emph{nicht} angelegt.\\
	\texttt{w}				& Schreiben -- Textmodus		& Dateicursor am Anfang der Datei.\newline Neuen Dateien werden angelegt, alte Dateien überschrieben.\\
	\texttt{wb}			& Schreiben -- Binärmodus	& Dateicursor am Anfang der Datei.\newline Neuen Dateien werden angelegt, alte Dateien überschrieben.\\
	\texttt{x}				& Schreiben -- Textmodus		& Dateicursor am Anfang der Datei.\newline Neuen Dateien werden angelegt, alte Dateien \emph{nicht} überschrieben.\\
	\texttt{xb}			& Schreiben -- Binärmodus	& Dateicursor am Anfang der Datei.\newline Neuen Dateien werden angelegt, alte Dateien \emph{nicht} überschrieben.\\
	\texttt{a}				& Anhängen -- Textmodus		& Dateicursor am \emph{Ende} der Datei. Neuen Dateien werden angelegt, alte Dateien \emph{nicht} überschrieben.\\
	\texttt{ab}			& Anhängen -- Binärmodus		& Dateicursor am \emph{Ende} der Datei. Neuen Dateien werden angelegt, alte Dateien \emph{nicht} überschrieben. \\
	\bottomrule
\end{tabular}
\end{center}
%
\end{frame}

% =========================================================================== %

\begin{frame}
%
\begin{hintbox}[Binär- oder Textmodus?]
\small
Ist die Datei in einem Texteditor lesbar?											\tabto{10.5cm} \Thus Textmodus

Sieht man im Textmodus nur kryptische Zeichen (wie \zB Bilder)	\tabto{10.5cm} \Thus Binärmodus
\end{hintbox}
%
\begin{hintbox}[Welcher Modus?]
\small
Lesen?				\tabto{3cm} \Thus \texttt{r} oder \texttt{rb}

Neu Anlegen?	\tabto{3cm} \Thus \texttt{w} oder \texttt{wb}

Ergänzen?		\tabto{3cm} \Thus \texttt{a} oder \texttt{ab}
\end{hintbox}
%
\begin{hintbox}[Pfadangaben -- Relativ und kompakt]
\small
Absolute Pfadangaben: unflexibel / schlecht portabel.

Nach Möglichkeit: Programme so konzipieren, dass notwendige Dateien in \emph{Unterordnern} liegen.
\end{hintbox}
%
\end{frame}

% =========================================================================== %

\begin{frame}[fragile]
%
\begin{columns}[T]
\column{.5\linewidth}
\begin{Large}
	{Dateien Schließen}
	\vspace{6pt}
\end{Large}
%
\begin{itemize}
\item Dateiverwaltung im Hintergrund sehr kompliziert
	\begin{itemize}
	\item Viele denkbare Szenarien: Festplatten, CDs, Netzwerk-Dateien, \ldots
	\item Verschiedene Dateisysteme (NTFS, FAT, ext, \ldots)
	\end{itemize}
\item Meiste Arbeit übernimmt das Betriebssystem
\item Für korrekte Arbeit aber: Wir müssen mitteilen, wann ein Zugriff beendet ist
\item[\Thus] Methode \inPy{close()}
\end{itemize}
%
\column{.5\linewidth}
\begin{codebox}[Syntax: Dateien Schließen]
\begin{minted}[fontsize=\scriptsize]{python}
handle.close()
\end{minted}
\end{codebox}
%
\begin{itemize}
\item Wenn vergessen: \idR automatisch geschlossen durch Python-Interpreter
\item Zeitversatz, bis Python das \enquote{bemerkt}
\end{itemize}
%
\begin{hintbox}[\texttt{open} und \texttt{close} in einem Schritt]
\small
Wenn Sie den Befehl \inPy{open} benutzen -- schreiben Sie direkt als nächstes das zugehörige \texttt{close}. Erst danach kümmern Sie sich um den Code dazwischen.
\end{hintbox}
\end{columns}
%
\end{frame}

% =========================================================================== %

\begin{frame}[fragile]{In Dateien Schreiben}
%
\begin{columns}[T]
\column{.5\linewidth}
\begin{itemize}
\item Sobald in einem Schreibmodus (\texttt{w}, \texttt{wb}, \texttt{a}, \ldots) geöffnet: Methode \texttt{write} steht zur Verfügung
\item Nimmt Entweder \inPy{str}ings als Parameter (Textmodus) oder \inPy{bytes}-Objekte (Binärmodus)
\item Wird so direkt in die Datei geschrieben
\item Zeilenumbruch wird \emph{nicht} automatisch angehängt.
\item Mehrere Schreibbefehle hintereinander: wird immer angehängt
\end{itemize}
%
\column{.5\linewidth}
\begin{codebox}[Beispiel: 2 Zeilen in Datei schreiben]
\begin{minted}[fontsize=\scriptsize]{python}
handle = open("output.txt", "w")

handle.write("Some text " * 2 + "\n")
handle.write( str([1, 2, 3]) )

handle.close()
\end{minted}
\end{codebox}
%
\begin{cmdbox}[Dateinhalt von \texttt{output.txt}]
\begin{minted}[fontsize=\scriptsize]{text}
Some text Some text
[1, 2, 3]
\end{minted}
\end{cmdbox}
\end{columns}
%
\end{frame}

% =========================================================================== %

\begin{frame}[fragile]
%
\begin{columns}[T]
\column{.5\linewidth}
\begin{Large}
	{Einschub: \inPy{bytes}-Objekt}
	\vspace{6pt}
\end{Large}
%
\begin{itemize}
\item Datencontainer, der Anordnung von Information im Arbeitsspeicher wiedergibt
\item Aus vielen Objekten generierbar
\item In viele Objekte umwandelbar
	\begin{itemize}
	\item Zu \inPy{int}: Direkt, über Methode \texttt{from\_bytes}
	\item Dazu \enquote{Byte order}: Lesen von links nach rechts oder von rechts nach links
	\item \enquote{big endian} oder \enquote{little endian}
	\item Zu \inPy{float}: Modul \texttt{struct}
	\end{itemize}
\item Kann im Binärmodus 1:1 geschrieben werden
\end{itemize}
%
\column{.5\linewidth}
\begin{codebox}[Beispiel: \texttt{bytes}]
\begin{minted}[fontsize=\scriptsize]{python}
data = [65, 66, 67, 68]
bdata = bytes(data)
print(bdata)
print(int.from_bytes(bdata, 'big'))

handle = open("output.dat", "wb")
handle.write(bdata)
handle.close()
\end{minted}
\end{codebox}
%
\begin{cmdbox}[Bildschirm-Ausgabe]
\begin{minted}[fontsize=\scriptsize]{text}
b'ABCD'
1094861636
\end{minted}
\end{cmdbox}
%
\begin{cmdbox}[Dateinhalt von \texttt{output.dat}]
\begin{minted}[fontsize=\scriptsize]{text}
ABCD
\end{minted}
\end{cmdbox}
\end{columns}
%
\end{frame}

% =========================================================================== %

\begin{frame}[fragile]
%
\begin{columns}[T]
\column{.5\linewidth}
\begin{Large}
	{Aus Dateien Lesen}
	\vspace{6pt}
\end{Large}
\begin{itemize}
\item Sobald in einem Lesemodus (\texttt{r} oder \texttt{rb}) geöffnet: Methode \texttt{read} steht zur Verfügung
\item Ein Optionaler Parameter
	\begin{itemize}
	\item Maximal zu lesende Schriftzeichen (Textmodus) oder Bytes
	\item Auslassen: Ganze Datei lesen
	\item Größer als Dateilänge -- ignorieren
	\item Kleiner als Dateilänge -- \enquote{Cursor} bleibt in Datei vor erstem ungelesenen Zeichen/Byte stehen
	\end{itemize}
\item Lesen, nachdem Dateiende erreicht wurde -- Fehlermeldung
\item Ergebnis: \inPy{str}ing- oder \inPy{bytes}-Variable
\end{itemize}
%
\column{.5\linewidth}
\begin{cmdbox}[Dateinhalt von \texttt{output.txt}]
\begin{minted}[fontsize=\scriptsize]{text}
Some text Some text
[1, 2, 3]
\end{minted}
\end{cmdbox}
%
\begin{codebox}[Beispiel: Textdateien Lesen]
\begin{minted}[fontsize=\scriptsize]{python}
handle = open("output.txt", "r")
firstWord = handle.read(4)
rest = handle.read()
handle.close()
print(firstWord)
print(rest)
\end{minted}
\end{codebox}
%
\begin{cmdbox}[Ausgabe]
\begin{minted}[fontsize=\scriptsize]{text}
Some
 text Some text
[1, 2, 3]
\end{minted}
\end{cmdbox}
%
\end{columns}
%
\end{frame}

% =========================================================================== %

\begin{frame}[fragile]{\texttt{readline} und \texttt{readlines}}
%
\begin{columns}[T]
\column{.5\linewidth}
\begin{itemize}
\item \texttt{readline}
	\begin{itemize}
	\item Methode für Datei-Handles im Text-Lesemodus
	\item Lies ganze Zeile ein
	\item Optionaler Parameter: Maximale Länge
	\item Zeilenumbruch wird mit eingelesen (falls nicht durch Parameter unterbunden)
	\end{itemize}
\item \texttt{readlines}
	\begin{itemize}
	\item Methode für Datei-Handles im Text-Lesemodus
	\item Lies \emph{alle} Zeilen ein, Speichere als \inPy{list}
	\item Optionaler Parameter: Maximale Zeilenzahl
	\item Zeilenumbruch wird mit eingelesen
	\end{itemize}
\end{itemize}
%
\column{.5\linewidth}
\begin{codebox}[Beispiel: \texttt{readline} und \texttt{readlines}]
\begin{minted}[fontsize=\scriptsize]{python}
handle = open("output.txt", "r")
print( handle.readline(), end="" )
print( handle.readline(), end="" )
handle.close()

handle = open("output.txt", "r")
data = handle.readlines()
print(data)
handle.close()
\end{minted}
\end{codebox}
%
\begin{cmdbox}[Ausgabe]
\begin{minted}[fontsize=\scriptsize]{text}
Some text Some text
[1, 2, 3]
['Some text Some text\n', '[1, 2, 3]\n']
\end{minted}
\end{cmdbox}
%
\end{columns}
%
\end{frame}

% =========================================================================== %

\begin{frame}[fragile]{Blockstruktur: \inPy{with .. as}}
%
\begin{columns}[T]
\column{.5\linewidth}
\begin{itemize}
\item Dateien automatisch schließen
\item Realisiert über Dunders \inPy{__enter__} und \inPy{__exit__}
	\begin{itemize}
	\item Dateihandle ist Klasseninstanz
	\item \inPy{with} ruft Methode \inPy{__enter__} auf, gibt Rückgabewert an \texttt{Handle}
	\item Am Ende der Einrückung: Automatisches Aurfufen von \inPy{__exit__}
	\item Im Fall von Dateihandles: \inPy{self.close()}
	\end{itemize}
\item Grundsätzlich zu empfehlen
\item Nur manchmal nicht möglich, wegen komplexer Programmstruktur
\end{itemize}
%
\column{.5\linewidth}
\begin{codebox}[Syntax: \texttt{with .. as}]
\begin{minted}[fontsize=\scriptsize]{python}
with open(Dateiname, Modus) as Handle :
    Anweisungen

# KEIN Handle.close()
\end{minted}
\end{codebox}
%
\begin{codebox}[Beispiel: Lesen Kompakt]
\begin{minted}[fontsize=\scriptsize]{python}
with open("output.txt", "r") as handle :
    data = handle.readlines()
print(data)
\end{minted}
\end{codebox}
%
\end{columns}
%
\end{frame}

% =========================================================================== %

\begin{frame}[fragile]{Blockstruktur: Dateihandles mit \inPy{for} und \inPy{list}}
%
\begin{columns}[T]
\column{.5\linewidth}
\begin{itemize}
\item Dateihandles sind \emph{Iterables}
\item Können mit \inPy{for} Zeile für Zeile durchgearbeitet werden
\item Können in eine \inPy{list} umgerechnet werden
\item Versteckte Zugriffe auf \texttt{readlines}
\item Vermutlich häufigster Gebrauchsfall für das Lesen von Dateien
\end{itemize}
%
\column{.5\linewidth}
\begin{codebox}[Beispiel: Drei Äquivalente Aufrufe]
\begin{minted}[fontsize=\scriptsize]{python}
with open("output.txt", "r") as handle :
    for line in handle :
        print(line)

with open("output.txt", "r") as handle :
    lines = list(handle)
    for line in lines :
        print(line)

with open("output.txt", "r") as handle :
    for line in handle.readlines() :
        print(line)
\end{minted}
\end{codebox}
%
\end{columns}
%
\end{frame}

% =========================================================================== %

\begin{frame}[fragile]{Dateicursor: \inPy{tell} und \inPy{seek}}
%
\begin{columns}[T]
\column{.5\linewidth}
\begin{itemize}
\item Beides: Methoden von Datei-Handles
\item	\texttt{tell}
	\begin{itemize}
	\item Gib aktuelle Dateicursor-Position aus
	\item Abstand vom Datei-Anfang in \enquote{Schritten}
	\item Binärmodi: Schritt = Bytes
	\item Textmodi: Schritt = Schriftzeichen
	\end{itemize}
\item \texttt{seek}
	\begin{itemize}
	\item Setze Cursorposition, selbe Regeln zu \enquote{Schritten}
	\item Parameter 1: Wie viele Schritte
	\item Parameter 2: Welcher Startpunkt
		\begin{itemize}
		\item 0: Dateianfang
		\item 1: Aktuelle Position
		\item 2: Dateiende
		\end{itemize}
	\end{itemize}
\end{itemize}
%
\column{.5\linewidth}
\begin{codebox}[Beispiel: \texttt{seek} und \texttt{tell}]
\begin{minted}[fontsize=\scriptsize]{python}
with open("output.txt", "r") as handle :
    text = handle.read(4)
    print( handle.tell() )
        # Ausgabe: 4
    
    handle.seek(-2, 1)
        # Zwei Zeichen zurück gehen
    print( handle.tell() )
        # Ausgabe: 2
    
    handle.seek(0, 2)
        # an das Dateiende springen
\end{minted}
\end{codebox}
%
\end{columns}
%
\end{frame}

% =========================================================================== %

\begin{frame}[fragile]{Pickle -- Python-Objekte Archivieren}
%
\begin{itemize}
\item Manche Berechnungen sind sehr aufwändig
\item Teilergebnisse für später wiederverwertbar, oder Berechnung in mehrere Sessions aufteilbar
\item Pickle: Modul, das Datenobjekte \emph{serialisiert}, \ie in einer zum Speichern oder Versenden geeignete Form bringt
\item Python-Eigenes Format, schlecht mit anderen Sprachen kompatibel
\item Dafür (im Prinzip) auf \emph{alles} in Python anwendbar
\item Braucht \inPy{import pickle}
\item Lesen und Schreiben im \emph{Binärmodus}
\item Schreiben: \inPy{pickle.dump(Objekt, Handle)}
\item Lesen: \inPy{Objekt = pickle.load(Handle)}
\end{itemize}
%
\end{frame}

% =========================================================================== %

\begin{frame}[fragile]
%
\begin{codebox}[Beispiel: Pickle -- Lesen und Schreiben]
\begin{minted}[linenos, fontsize=\scriptsize]{python}
import pickle

class Foo :
    def __init__(self) :
        self.bar = 1 + 1j
    def __str__(self) :
        return "Foo object"

complexNumber = -0.3 + 4.1j
classObject   = Foo()

with open("archive.pkl", "wb") as handle :
    pickle.dump(complexNumber, handle)
    pickle.dump(classObject  , handle)
with open("archive.pkl", "rb") as handle :
    readComplex     = pickle.load(handle)
    readClassObject = pickle.load(handle)

print(readComplex)
print(readClassObject, readClassObject.bar)
\end{minted}
\end{codebox}
%
\end{frame}

% =========================================================================== %

\begin{frame}
%
\begin{warnbox}[Sicherheitslücke Pickle]
Pickle kann alles serialisieren -- auch Funktionen und ganze Module. Das kann praktisch sein, erlaubt aber auch, Schadcode in einem Archiv unterzubringen. Aus der Datei selbst ist schwer ersichtlich, was geladen wird.

Laden Sie nur Pickles, denen Sie sicher vertrauen!
\end{warnbox}
%
\end{frame}

% =========================================================================== %

\begin{frame}[fragile]{JSON -- Portable Archivierung}
%
\begin{itemize}
\item \emph{JavaScript Object Notation}
\item Alternativer Ansatz zu Pickle, selbes Ziel
\item Daten werden in lesbarem Klartext gespeichert
\item Unterstützt nur einige Datentypen: \inPy{int}, \inPy{float}, \inPy{str}, \inPy{bool}, \inPy{list}, \inPy{tuple}, \inPy{dict}, \inPy{None}
\item Andere Datentypen: Manuell dekonstruieren
\item Dafür gut kompatibel mit anderen Sprachen
\item Logik ähnlich wie mit Pickle:
	\begin{itemize}
	\item \inPy{import json}
	\item Öffnen im \emph{Text}modus
	\item Schreiben: \inPy{json.dump(Objekt, Handle)}
	\item Lesen: \inPy{Objekt = json.load(Handle)}
	\item Mehrere Objekte in einem \inPy{dict} zusammenfassen
	\end{itemize}
\end{itemize}
%
\end{frame}

% =========================================================================== %

\begin{frame}[fragile]
%
\begin{codebox}[Beispiel: JSON -- Lesen und Schreiben]
\begin{minted}[linenos, fontsize=\scriptsize]{python}
import json

complexNumber = -0.3 + 4.1j
realPart      = complexNumber.real
imaginaryPart = complexNumber.imag
dictionary  = {1 : "one", 2 : "two", 3 : "three", 4 : ["list", "of", "strings"]}

with open("archive.json", "w") as handle :
    json.dump(
        {"real" : realPart,
         "imag" : imaginaryPart,
         "dict" : dictionary},
        handle
    )
with open("archive.json", "r") as handle :
    jsonObjects = json.load(handle)
readComplex = complex(jsonObjects["real"], jsonObjects["imag"])

print(readComplex)
print(jsonObjects["dict"])
\end{minted}
\end{codebox}
%
\end{frame}

% =========================================================================== %

\begin{frame}[fragile]{CSV -- Tabellen Lesen und Schreiben}
%
\begin{itemize}
\item \emph{Comma Separated Values}. Von Excel \ua Tabellenkalkulationsprogrammen verwendet
\item \inPy{str}ings, \inPy{int}s und \inPy{float}s in Spalten und Zeilen
\item Spalten durch verschiedene Zeichen abgetrennt, klassischwerweise Komma, oft aber auch Leerzeichen oder Tabulatoren
\item Doppeltes Trennzeichen \Thus Leerer Eintrag
\item \inPy{str}ings durch \emph{Quote Chars} eingerahmt, üblicherweise \texttt{"}doppelte Anführungszeichen\texttt{"}
\end{itemize}
%
\end{frame}

% =========================================================================== %

\begin{frame}[fragile]{CSV -- Tabellen Lesen und Schreiben}
%
\begin{itemize}
\item Braucht \inPy{import csv}
\item \texttt{csv.reader}: Klasse, die eine CSV-Datei nach bestimmten Regeln ausliest.
\item Parameter:
	\begin{itemize}
	\item \texttt{handle} -- Dateihandle, geöffnet im Text-Lesemodus (\inPy{"r"})
	\item Optional: \texttt{delimiter} -- Zeichen, das Spalten voneinander trennt (default: \inPy{","})
	\item Optional: \texttt{quotechar} -- Zeichen, das Strings einrahmt
	\end{itemize}
\item \texttt{csv.reader} kann mit \inPy{for} durchlaufen werden
	\begin{itemize}
	\item Elemente sind dann \inPy{list}s, die die Zeilen der CSV darstellen
	\item Elemente der \inPy{list}: Spalten der aktuellen Zeile
	\end{itemize}
\item Weitere Details: \url{https://docs.python.org/3/library/csv.html}
\end{itemize}
%
\end{frame}

% =========================================================================== %

\begin{frame}[fragile]
%
\begin{tcbraster}[raster columns=2,
                  raster equal height,
                  nobeforeafter,
                  raster column skip=0.5cm]
\begin{cmdbox}[Zu lesende Datei]
\begin{minted}[fontsize=\scriptsize, obeytabs, tabsize=14]{text}
#time in s	voltage in V
0	0
1	0.1
2	1.6
3	2.9
4	10.1
5	11.5
6	9.8
7	3.1
8	0.2
9	0
\end{minted}
\end{cmdbox}
%
\begin{codebox}[Beispiel: Datei einlesen]
\begin{minted}[fontsize=\scriptsize, linenos]{python}
import csv

with open("tab.txt", "r") as handle :
    reader = csv.reader(
        handle,
        delimiter='\t'
    )
    for row in reader :
        print(row)
\end{minted}
\end{codebox}
\end{tcbraster}
%
\begin{cmdbox}[Ausgabe]
\begin{minted}[fontsize=\scriptsize, obeytabs, tabsize=14]{text}
['#time in s', 'voltage in V']
['0', '0']
['1', '0.1']
['2', '1.6']
...
\end{minted}
\end{cmdbox}
%
\end{frame}

% =========================================================================== %

\begin{frame}[fragile]{Pickle, JSON, CSV -- Wann Was Benutzen?}
%
\begin{tcbraster}[raster columns=3,
                  raster equal height,
                  nobeforeafter,
                  raster column skip=0.5cm]
\begin{tcolorbox}[title=Pickle]
\begin{itemize}
\item Sehr komplexe Objekte
\item Reine Python-Projekte
\item Keine Weitergabe vorgesehen
\item Muss nicht für Menschen lesbar sein
\end{itemize}
\end{tcolorbox}
%
\begin{tcolorbox}[title=JSON]
\begin{itemize}
\item Mittlere Komplexität der Objekte
\item Keine geordnete Struktur
\item Klartext-Lesbarkeit
\item Austausch mit anderen Programmen
\end{itemize}
\end{tcolorbox}
%
\begin{tcolorbox}[title=CSV]
\begin{itemize}
\item Tabellenstruktur
\item Sehr einfache Werte-Typen in Zeilen und Spalten
\item Austausch mit anderen Programmen, insbesondere Tabellenkalkulationsprogramme
\end{itemize}
\end{tcolorbox}
\end{tcbraster}
%
\end{frame}